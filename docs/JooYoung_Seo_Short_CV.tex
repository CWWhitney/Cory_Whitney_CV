%!TEX TS-program = xelatex
%!TEX encoding = UTF-8 Unicode
% Awesome CV LaTeX Template for CV/Resume
%
% This template has been downloaded from:
% https://github.com/posquit0/Awesome-CV
%
% Author:
% Claud D. Park <posquit0.bj@gmail.com>
% http://www.posquit0.com
%
%
% Adapted to be an Rmarkdown template by Mitchell O'Hara-Wild
% 23 November 2018
%
% Template license:
% CC BY-SA 4.0 (https://creativecommons.org/licenses/by-sa/4.0/)
%
%-------------------------------------------------------------------------------
% CONFIGURATIONS
%-------------------------------------------------------------------------------
% A4 paper size by default, use 'letterpaper' for US letter
\documentclass[11pt, a4paper]{awesome-cv}

% Configure page margins with geometry
\geometry{left=1.4cm, top=.8cm, right=1.4cm, bottom=1.8cm, footskip=.5cm}

% Specify the location of the included fonts
\fontdir[fonts/]

% Color for highlights
% Awesome Colors: awesome-emerald, awesome-skyblue, awesome-red, awesome-pink, awesome-orange
%                 awesome-nephritis, awesome-concrete, awesome-darknight

\colorlet{awesome}{awesome-red}

% Colors for text
% Uncomment if you would like to specify your own color
% \definecolor{darktext}{HTML}{414141}
% \definecolor{text}{HTML}{333333}
% \definecolor{graytext}{HTML}{5D5D5D}
% \definecolor{lighttext}{HTML}{999999}

% Set false if you don't want to highlight section with awesome color
\setbool{acvSectionColorHighlight}{true}

% If you would like to change the social information separator from a pipe (|) to something else
\renewcommand{\acvHeaderSocialSep}{\quad\textbar\quad}

\def\endfirstpage{\newpage}

%-------------------------------------------------------------------------------
%	PERSONAL INFORMATION
%	Comment any of the lines below if they are not required
%-------------------------------------------------------------------------------
% Available options: circle|rectangle,edge/noedge,left/right

\name{JooYoung}{Seo}

\position{RStudio Intern for Summer 2020; Ph.D.~Candidate (ABD)}
\address{Learning, Design, and Technology, 301 Keller Building, University Park, PA 16802, USA}

\mobile{+1 814-777-5825}
\email{\href{mailto:jooyoung@psu.edu}{\nolinkurl{jooyoung@psu.edu}}}
\homepage{jooyoungseo.com}
\github{jooyoungseo}
\linkedin{jooyoungseo}
\twitter{sjystu}

% \gitlab{gitlab-id}
% \stackoverflow{SO-id}{SO-name}
% \skype{skype-id}
% \reddit{reddit-id}

\quote{I am a learning scientist, software engineer, and internationally-certified accessibility professional.}

\usepackage{booktabs}

% Templates for detailed entries
% Arguments: what when with where why
\usepackage{etoolbox}
\def\detaileditem#1#2#3#4#5{%
\cventry{#1}{#3}{#4}{#2}{\ifx#5\empty\else{\begin{cvitems}#5\end{cvitems}}\fi}\ifx#5\empty{\vspace{-4.0mm}}\else\fi}
\def\detailedsection#1{\begin{cventries}#1\end{cventries}}

% Templates for brief entries
% Arguments: what when with
\def\briefitem#1#2#3{\cvhonor{}{#1}{#3}{#2}}
\def\briefsection#1{\begin{cvhonors}#1\end{cvhonors}}

\providecommand{\tightlist}{%
	\setlength{\itemsep}{0pt}\setlength{\parskip}{0pt}}

%------------------------------------------------------------------------------




\begin{document}

% Print the header with above personal informations
% Give optional argument to change alignment(C: center, L: left, R: right)
\makecvheader

% Print the footer with 3 arguments(<left>, <center>, <right>)
% Leave any of these blank if they are not needed
% 2019-02-14 Chris Umphlett - add flexibility to the document name in footer, rather than have it be static Curriculum Vitae
\makecvfooter
  {March 05, 2020}
    {JooYoung Seo~~~·~~~Curriculum Vitae}
  {\thepage}


%-------------------------------------------------------------------------------
%	CV/RESUME CONTENT
%	Each section is imported separately, open each file in turn to modify content
%------------------------------------------------------------------------------



\hypertarget{education}{%
\section{Education}\label{education}}

\detailedsection{\detaileditem{Ph.D. in Learning, Design, and Technology}{Aug. 2016 - Present}{The Pennsylvania State University}{University Park, PA}{\item{Expected graduation: May 2021.}\item{Dissertation Title: ``Discovering Informal Learning Cultures of Blind Individuals Pursuing STEM Disciplines: A Quantitative Ethnography Using Public Listserv Archives.''}\item{Committee members: Drs. Gabriela T. Richard (adviser; dissertation chair), Roy B. Clariana, ChanMin Kim, and Mary Beth Rosson.}}\detaileditem{M.Ed. in Learning, Design, and Technology}{Aug. 2014 - May. 2016}{The Pennsylvania State University}{University Park, PA}{\item{GPA 3.97/4.0.}}\detaileditem{B.A. in Education}{Mar. 2009 - Feb. 2014}{Sungkyunkwan University}{Seoul, South Korea}{\item{GPA 4.08/4.50.}}\detaileditem{B.A. in English Literature}{Mar. 2009 - Feb. 2014}{Sungkyunkwan University}{Seoul, South Korea}{\item{GPA 4.08/4.50.}}}

\hypertarget{working-experience}{%
\section{Working Experience}\label{working-experience}}

\detailedsection{\detaileditem{Summer Intern}{May. 2020 - Aug. 2020}{RStudio}{Redmond, WA}{\empty}\detaileditem{}{}{}{}{\empty}\detaileditem{Co-Founder and Project Manager}{Jul. 2010 - Jun. 2011}{ICE Soft}{Seoul, South Korea}{\empty}\detaileditem{Intern}{May. 2010 - Nov. 2010}{National Human Rights Commission of Korea}{Seoul, South Korea}{\empty}}

\hypertarget{research-experience}{%
\section{Research Experience}\label{research-experience}}

\detailedsection{\detaileditem{Graduate Assistant}{Aug. 2016 - Present}{IT Accessibility Team, Teaching and Learning with Technology (TLT)}{University Park, PA}{\empty}\detaileditem{}{}{}{}{\empty}\detaileditem{Graduate Researcher}{Aug. 2016 - Present}{Playful Learning and Inclusive Design Research Group}{University Park, PA}{\empty}\detaileditem{Principal Investigator}{May. 2019 - Present}{Online Interactions of the Blind Research Project}{University Park, PA}{\empty}\detaileditem{Project Manager}{May. 2018 - Present}{Accessible RMarkdown Online Writer (AROW) Project, Teaching and Learning with Technology (TLT)}{University Park, PA}{\empty}\detaileditem{Principal Investigator}{Jun. 2017 - Present}{Accessibility of Maker Toolkits Research Project}{University Park, PA}{\empty}\detaileditem{Site Manager}{Feb. 2015 - Dec. 2015}{Learning2CC Project}{University Park, PA}{\empty}\detaileditem{Graduate Researcher}{Sep. 2014 - May. 2016}{Avenue PM Research Group}{University Park, PA}{\empty}\detaileditem{Student Researcher}{Feb. 2012 - Mar. 2012}{The Ministry of Knowledge Economy and Seoul National University}{Seoul, South Korea}{\empty}}

\hypertarget{teaching-experience}{%
\section{Teaching Experience}\label{teaching-experience}}

\detailedsection{\detaileditem{Accessibility Instructor}{Mar. 2015 - Present}{IT Accessibility Team, Teaching and Learning with Technology (TLT)}{University Park, PA}{\empty}\detaileditem{Assistant English Teacher}{Mar. 2013 - May. 2013}{Gunpo High School}{Gyeonggi, South Korea}{\empty}}

\hypertarget{publications}{%
\section{Publications}\label{publications}}

\hypertarget{refereed-journal-papers}{%
\subsection{Refereed Journal Papers}\label{refereed-journal-papers}}

\begingroup
\setlength{\parindent}{-0.5in}
\setlength{\leftskip}{0.5in}

\hypertarget{refs_journals}{}
\leavevmode\hypertarget{ref-seo2019maker}{}%
\textbf{Seo, J.} (2019). Is the maker movement inclusive of ANYONE?: Three accessibility considerations to invite blind makers to the making world. \emph{TechTrends}, \emph{63}(5), 514--520. \url{https://doi.org/10.1007/s11528-019-00377-3}

\leavevmode\hypertarget{ref-seo2019arow}{}%
\textbf{Seo, J.}, \& McCurry, S. (2019). LaTeX is not easy: Creating accessible scientific documents with r markdown. \emph{Journal on Technology and Persons with Disabilities}, \emph{7}, 157--171.

\endgroup

\hypertarget{papers-in-refereed-conference-proceedings}{%
\subsection{Papers in Refereed Conference Proceedings}\label{papers-in-refereed-conference-proceedings}}

\begingroup
\setlength{\parindent}{-0.5in}
\setlength{\leftskip}{0.5in}

\hypertarget{refs_proceedings}{}
\leavevmode\hypertarget{ref-seo2020coding}{}%
\textbf{Seo, J.}, \& Richard, G. T. (in press-). Coding through touch: Exploring and re-designing tactile making activities with learners with visual dis/abilities. In M. Gresalfi \& I. Horn (Eds.), \emph{Interdisciplinarity in the learning sciences, 14th international conference of the learning sciences (icls) 2020}. International Society of the Learning Sciences (ISLS).

\leavevmode\hypertarget{ref-seo2018making}{}%
\textbf{Seo, J.} (2018). Accessibility and inclusivity in making: Engaging learners with all abilities in making activities. In \emph{Proceedings of the 3rd learning sciences graduate student conference} (pp. 141--142). LSGSC Planning Team.

\leavevmode\hypertarget{ref-seo2018accessibility}{}%
\textbf{Seo, J.}, \& Richard, G. T. (2018). Accessibility, making and tactile robotics: Facilitating collaborative learning and computational thinking for learners with visual impairments. In J. Kay \& R. Luckin (Eds.), \emph{Rethinking learning in the digital age: Making the learning sciences count, 13th international conference of the learning sciences (icls) 2018} (Vol. 3, pp. 1755--1757). International Society of the Learning Sciences (ISLS).

\leavevmode\hypertarget{ref-konecki2017role}{}%
Konecki, M., Lovrenčić, S., \textbf{Seo, J.}, \& LaPierre, C. (2017). The role of ict in aiding visually impaired students and professionals. \emph{Proceedings of the 11th Multidisciplinary Academic Conference}, 148.

\leavevmode\hypertarget{ref-seo2017embracing}{}%
\textbf{Seo, J.}, AlQahtani, M., Ouyang, X., \& Borge, M. (2017). Embracing learners with visual impairments in cscl. In B. K. Smith, M. Borge, E. Mercier, \& K. Y. Lim (Eds.), \emph{Making a difference: Prioritizing equity and access in cscl, 12th international conference on computer supported collaborative learning (cscl) 2017} (Vol. 2). International Society of the Learning Sciences (ISLS).

\leavevmode\hypertarget{ref-seo2019discovering}{}%
\textbf{Seo, J.} (2019). Discovering informal learning cultures of blind individuals pursuing stem disciplines: A quantitative ethnography using listserv archives. \emph{The First International Conference on Quantitative Ethnography: Doctoral Consortium}, S66--S67.

\endgroup

\hypertarget{presentations}{%
\section{Presentations}\label{presentations}}

\hypertarget{peer-reviewed-conference-presentations}{%
\subsection{Peer-Reviewed Conference Presentations}\label{peer-reviewed-conference-presentations}}

\begingroup
\setlength{\parindent}{-0.5in}
\setlength{\leftskip}{0.5in}

\textbf{Seo, J.}, \& Richard, G. T. (2020, June). \emph{Coding through touch: exploring and re-designing tactile making activities with learners with visual dis/abilities}. Paper will be presented at the 14th International Conference of the Learning Sciences (ICLS), Nashville, TN.

\textbf{Seo, J.}, \& Richard, G. T. (2020, April). \emph{Maker inclusivity = maker accessibility: further interrogations for diverse participation}. Poster will be presented at the annual meeting of the American Educational Research Association (AERA), San Francisco, CA.

\textbf{Seo, J.} (2019, October). \emph{Discovering informal learning cultures of blind individuals pursuing STEM disciplines: a quantitative ethnography using listserv archives}. Poster presented at the Doctoral Consortium of the first International Conference on Quantitative Ethnography (ICQE), Madison, WI. \emph{Awarded the best doctoral proposal fellowship}.

\textbf{Seo, J.}, \& McCurry, S. (2019, March). \emph{LaTeX is NOT easy: creating accessible scientific documents with R markdown}. Paper presented at the annual meeting of the CSUN Assistive Technology Conference, Northridge, CA.

\textbf{Seo, J.} (2018, October). \emph{Accessibility and inclusivity in making: engaging learners with all abilities in making activities}. Paper presented at the annual meeting of the Learning Sciences Graduate Student Conference (LSGSC), Nashville, TN.

\textbf{Seo, J.}, \& Richard, G. T. (2018, June). \emph{Accessibility, making and tactile robotics: facilitating collaborative learning and computational thinking for learners with visual impairments}. Poster presented at the 13th International Conference of the Learning Sciences (ICLS), London, UK.

\textbf{Seo, J.}, \& Richard, G. T. (2018, April). \emph{Furthering inclusivity in making: a framework for accessible design of makerspaces for learners with disabilities}. Poster presented at the annual meeting of the American Educational Research Association (AERA), New York City, NY.

Bunag, T., Aniela, L., Nielsen, M. C., \& \textbf{Seo, J.} (2017, November). \emph{The rapidly changing world of accessible online learning}. Presented at the Panel Discussion: DDL - Accessible Online Learning In Concurrent Presentation of the Association for Educational Communications and Technology (AECT), Jacsonville, FL.

\textbf{Seo, J.} (2017, September). \emph{Tactile access to visualized statistical data using R}. Poster presented at the annual meeting of the ACM Richard Tapia Celebration of Diversity in Computing, Atlanta, GA.

\textbf{Seo, J.}, AlQahtani, M., Ouyang, X., \& Borge, M. (2017, June). \emph{Embracing learners with visual impairments in CSCL}. Paper presented at the 12th International Conference on Computer Supported Collaborative Learning (CSCL), Philadelphia, PA.

Liao, J., Patcyk, M., \textbf{Seo, J.}, \& Hooper, S. (2016, October). \emph{Using hierarchical linear modeling to measure growth rate in a gamified CBM environment}. Paper presented at the annual meeting of the Northeastern Educational Research Association (NERA), Trumbull, CT.

\textbf{Seo, J.} (2016, March). \emph{Engaging blind learners in statistics study using R}. Presented at the annual meeting of the Teaching and Learning with Technology (TLT) Symposium, University Park, PA.

Kim, K., \textbf{Seo, J.}, \& Clariana, R. B. (2016, March). \emph{Automatic knowledge structure measure in online courses}. Presented at the annual meeting of the Teaching and Learning with Technology (TLT) Symposium, University Park, PA.

\textbf{Seo, J.} (2015, November). \emph{Assistive technologies for equal access in general education}. Presented at the annual meeting of the Association for Educational Communications and Technology (AECT), Indianapolis, IN.

\textbf{Seo, J.}, \& Park, E. (2015, October). \emph{The more accessible, the more potential: simple tips for online accessibility}. Presented at the Technology and Learning Conference, Blue Bell, PA.

\endgroup

\hypertarget{software-developments-and-publications}{%
\section{Software Developments and Publications}\label{software-developments-and-publications}}

\hypertarget{data-science-packages-in-comprehensive-r-archive-network-cran}{%
\subsection{Data Science Packages in Comprehensive R Archive Network (CRAN)}\label{data-science-packages-in-comprehensive-r-archive-network-cran}}

\begingroup
\setlength{\parindent}{-0.5in}
\setlength{\leftskip}{0.5in}

\hypertarget{refs_R_packages}{}
\leavevmode\hypertarget{ref-R-youtubecaption}{}%
\textbf{Seo, J.}, \& Choi, S. (2020). \emph{Youtubecaption: Downloading youtube subtitle transcription in a tidy tibble data frame}. \url{https://CRAN.R-project.org/package=youtubecaption}

\leavevmode\hypertarget{ref-R-ezpickr}{}%
\textbf{Seo, J.}, \& Choi, S. (2019). \emph{Ezpickr: Easy data import using gui file picker and seamless communication between an excel and r}. \url{https://CRAN.R-project.org/package=ezpickr}

\leavevmode\hypertarget{ref-R-mboxr}{}%
\textbf{Seo, J.}, \& Choi, S. (2019). \emph{Mboxr: Reading, extracting, and converting an mbox file into a tibble}. \url{https://CRAN.R-project.org/package=mboxr}

\endgroup

\hypertarget{open-source-project-on-github}{%
\subsection{Open-Source Project on GitHub}\label{open-source-project-on-github}}

\begingroup
\setlength{\parindent}{-0.5in}
\setlength{\leftskip}{0.5in}

\hypertarget{refs_github_projects}{}
\leavevmode\hypertarget{ref-R-edmdown}{}%
\textbf{Seo, J.} (2019). \emph{Edmdown: Writing a reproducible article for journal of educational datamining in r markdown}. \url{https://github.com/jooyoungseo/edmdown}

\leavevmode\hypertarget{ref-R-ezviewr}{}%
\textbf{Seo, J.} (2019). \emph{Ezviewr: View tidy data in preferable spreadsheet}. \url{https://github.com/jooyoungseo/ezviewr}

\leavevmode\hypertarget{ref-R-jladown}{}%
\textbf{Seo, J.} (2019). \emph{Jladown: Writing a reproducible article for journal of learning analyticsin r markdown}. \url{https://github.com/jooyoungseo/jladown}

\leavevmode\hypertarget{ref-R-tactileR}{}%
\textbf{Seo, J.} (2019). \emph{TactileR: Converting r graphics into a braille ready-to-print pdf}. \url{https://github.com/jooyoungseo/tactileR}

\leavevmode\hypertarget{ref-webrender}{}%
\textbf{Seo, J.}, \& McCurry, S. (2019). \emph{AROW: Accessible rmarkdown online writer}. \url{http://www.arowtool.com}

\endgroup

\hypertarget{officially-code_contributing-r-packages}{%
\subsection{Officially Code\_Contributing R Packages}\label{officially-code_contributing-r-packages}}

\begin{itemize}
\tightlist
\item
  My \href{https://github.com/pulls?q=is\%3Apr+author\%3Ajooyoungseo+archived\%3Afalse+is\%3Aclosed}{18 GitHub pull requests} have been peer-reviewed and merged for active data science R packages:
\end{itemize}

\detailedsection{\detaileditem{rmarkdown: Dynamic Documents for R}{2019}{Allaire, J., Xie, Y., et al.}{https://cran.r-project.org/web/packages/rmarkdown/}{\item{Contributed to improving ioslides\_presentation accessibility for screen reading software.}}\detaileditem{bookdown: Authoring Books and Technical Documents with R Markdown}{2019}{Xie, Y.}{https://cran.r-project.org/web/packages/bookdown/}{\item{Developed context\_document2, github\_document2, beamer\_presentation2, html\_fragment2, html\_notebook2, html\_vignette2, ioslides\_presentation2, slidy\_presentation2 and rtf\_document2 functions to enable R markdown users to utilize cross-references for figures and tables through bookdown package.}}\detaileditem{distill: 'R Markdown' Format for Scientific and Technical Writing}{2019}{Allaire, J., Iannone, R., \& Xie, Y.}{https://cran.r-project.org/web/packages/distill/}{\item{Contributed to displaying more metadata for journal article bib\_entries.}\item{Improved screen reader accessibility for navigation bar.}\item{Added missing Google Scholar meta tags for conference, technical report, and dissertation types.}}\detaileditem{wordcountaddin: Word counts and readability statistics in R markdown documents}{2019}{Marwick, B.}{https://github.com/benmarwick/wordcountaddin}{\item{Contributed text\_stats() function for R markdown users to Get a word count and some other stats for selected text (excluding code chunks and inline code).}}\detaileditem{BrailleR: Improved Access for Blind Users}{2019}{Godfrey, A. J. R., Warren, D., Murrell, P., Bilton, T., \& Sorge, V.}{https://cran.r-project.org/web/packages/BrailleR/}{\item{Contributed BRLThis() function to enable blind R users to convert a graph into a pdf ready for embossing in Braille.}}}

\hypertarget{awards-and-honors}{%
\section{Awards and Honors}\label{awards-and-honors}}

\detailedsection{\detaileditem{AccessSTEM Resume contest winners Award}{Dec. 2019}{AccessComputing, University of Washington}{Seattle, WA}{\item{Sponsored by The National Science Foundation (NSF).}}\detaileditem{Best Doctoral Proposal Fellowship}{Oct. 2019}{The First International Conference on Quantitative Ethnography}{Madison, WI}{\item{Sponsored by Cengage.}}\detaileditem{Andrew V. Kozak Memorial Fellowship}{2019}{The PDK Educational Foundation}{University Park, PA}{\item{1500}}\detaileditem{AERA Pre-Conference Travel Award}{Apr. 2019}{The Spencer Foundation}{Toronto, ON, Canada}{\empty}\detaileditem{Delta Gamma Golden Anchor Award}{2017}{The Pennsylvania State University}{University Park, PA}{\empty}\detaileditem{Delta Gamma Golden Anchor Award}{2015}{The Pennsylvania State University}{University Park, PA}{\empty}\detaileditem{ACM Richard Tapia Celebration of Diversity in Computing Conference Scholarship}{Sep. 2017}{The National Science Foundation (NSF)}{Atlanta, GA}{\item{Included free conference registration, shared hotel accommodations and a \$500 travel stipend.}}\detaileditem{ACM Student Research Competition Travel Award}{Sep. 2017}{Microsoft Research}{Atlanta, GA}{\item{\$500 travel stipend was awarded for ACM TAPIA 2017 Student Research Competition (SRC).}}\detaileditem{CSUN Conference Travel Award}{Mar. 2017}{AccessComputing, University of Washington}{Seattle, WA}{\item{Sponsored by The National Science Foundation (NSF).}}\detaileditem{Summer Tuition Assistance Funding}{2017}{College of Education}{University Park, PA}{\empty}\detaileditem{Graduate Student Travel Grant}{Aug. 2016 - Present}{College of Education}{University Park, PA}{\empty}\detaileditem{Graduate Assistantship}{Aug. 2016 - Present}{Teaching and Learning with Technology}{University Park, PA}{\empty}\detaileditem{Outstanding Academic Award}{2014 - 2019}{Hyomyoung Scholarship}{Seoul, South Korea}{\empty}\detaileditem{Academic Scholarship for Doctoral Research}{2016 - 2018}{ChungInwook Scholarship Foundation}{Seoul, South Korea}{\empty}\detaileditem{All Nations Church Graduate Scholarship}{2016}{ANC Scholarship Foundation}{Lake View Terrace, CA}{\empty}\detaileditem{Future Interdisciplinary Study of Global Korea Scholarship for Master’s Research}{2014 - 2016}{National Institute for International Education, Korean Government}{Seoul, South Korea}{\empty}\detaileditem{Academic Achievement Scholarship}{2009 - 2013}{Sungkyunkwan University}{Seoul, South Korea}{\empty}}

\hypertarget{grants}{%
\section{Grants}\label{grants}}

\detailedsection{\detaileditem{Dissertation Research Initiation Grant}{Feb. 2020 - Dec. 2020}{College of Education, The Pennsylvania State University}{University Park, PA}{\item{Selected as one of the 10 outstanding dissertation research proposals in 2019-2020 academic year.}\item{Funding amount: \$600.}}\detaileditem{Providing Touchable Knowledge Structure Graphic Feedback for Blind Online Learners: Tablet-Based Haptic Feedback and Paper-Based Tactile Feedback}{Mar. 2016 - Aug. 2017}{Center for Online Innovation in Learning, The Pennsylvania State University}{University Park, PA}{\item{Co-led (Co-PI) a development project funded by Penn State to provide blind online learners with accessible network graphs that measure students’ knowledge structure.}\item{Developed a prototype website and mobile app for accessible knowledge structure.}\item{Funding amount: \$33,060.}}\detaileditem{ICE Soft: I See Everything through Assistive Technology}{Jul. 2010 - Jun. 2011}{Seed Funding for Start-Up Company for the Youth, City of Seoul}{Seoul, South Korea}{\item{Co-Founded and Project-Managed ICE Soft.}\item{Worked with a team and developed an gps app for the blind.}\item{Funding amount: \$30,000 USD.}}}

\hypertarget{invited-guest-lectures}{%
\subsection{Invited Guest Lectures}\label{invited-guest-lectures}}

\detailedsection{\detaileditem{Blockers for Users on a Screen Reader}{Oct. 2019}{EIT Accessibility Group, The Pennsylvania State University}{University Park, PA}{\item{Invited guest talk to a webinar to train Penn State instructional designers for key WCAG 2.1 guidelines to make content more accessible including: image alt text, clear link text and heading structure, proper table structure, form and button labels, the need for keyboard functionality and how to convey information regardless of visual formatting.}}\detaileditem{A Small Step to Take Your Data Analysis to Another Level}{Jul. 2019}{Chonnam National University}{Gwangju, South Korea}{\item{Invited guest talk to CNU to teach nursing faculty and students for basic concept of machine learning, computer-assisted text mining, and topic modelling to improve their qualitative data reliability.}}\detaileditem{Being a Reasonable Realist: Wise Negotiation between Give and Take}{Jun. 2019}{Sungkyunkwan University}{Seoul, South Korea}{\item{Invited guest talk to SKKU “Student Success Center” as one of the successful role models to inspire undergraduate students for future planning.}}\detaileditem{Accessibility of Math, Statistics, and Social Sciences}{May. 2019}{The MathML Meeting, The Pennsylvania State University}{University Park, PA}{\item{Invited guest talk to Penn State MathML group to present assistive technology and accessibility for STEM content.}}\detaileditem{Key-Note Speech for STEM Extension}{Jul. 2018}{BBVS Summer Academy STEM Extension}{University Park, PA}{\item{Invited key-note speaker to the STEM (Science, Technology, Engineering, and Mathematics) week of “The Summer Academy for Students who are Blind or Visually Impaired.”}\item{Hosted by the Pennsylvania Department of Labor and Industry, Office of Vocational Rehabilitation’s Bureau of Blindness and Visual Services, in partnership with the Pennsylvania Department of Education, Bureau Of Special Education’s Pennsylvania Training and Technical Assistance Network and Pennsylvania State University’s College of Education and College of Health and Human Development.}}\detaileditem{Adaptive Technology Lesson}{Apr. 2018}{LDT 100 - ``World Technologies and Learning'', The Pennsylvania State University}{University Park, PA}{\item{Two times invited guest talk to Dr. Joshua Kirby's class for the week of “The Cost of 21st Century Education.”}}\detaileditem{Universal Design 101: Three Fundamental Frameworks for an Equitable World}{Apr. 2017}{LDT First Friday Speaking Series, The Pennsylvania State University}{University Park, PA}{\item{Invited guest talk to the Learning, Design, and Technology (LDT) program of Penn State to introduce theoretical and practical background of Universal Design for Learning.}}\detaileditem{Inclusive Making}{Oct. 2017}{Northwestern University}{Evanston, IL}{\item{Invited guest talk to Dr. Marcelo Worsley’s “Inclusive Making” class for Learning Sciences Program.}}\detaileditem{Accessibility Testing Using NVDA}{Jul. 2017}{The Accessibility Users Group, The Pennsylvania State University}{University Park, PA}{\item{Invited guest talk to the Penn State online learning Accessibility User Group to train web accessibility testing with open-source screen reader NVDA.}}\detaileditem{Non-Visual Access to Canvas with Assistive Technology}{Mar. 2017}{Canvas Day 17}{University Park, PA}{\item{Invited guest to a showcase of Canvas, a learning management tool, to demonstrate how to interact with the online system using assistive technology for students with disabilities.}}\detaileditem{Student Panel Discussion: Student Issues}{Mar. 2017}{Ed-ICT International Network: Disabled students, ICT, post-compulsory education \& employment}{Seattle, WA}{\item{Invited student panel to the first Ed-ICT International Network symposium.}}\detaileditem{Accessibility: The First Step towards Ability}{Jul. 2016}{Korea Employment Promotion Agency for the Disabled}{Gyeonggi, South Korea}{\item{Invited guest talk to KEAD to train the employees in concept of accessibility and universal design.}}}

\end{document}
