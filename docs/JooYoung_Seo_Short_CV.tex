%!TEX TS-program = xelatex
%!TEX encoding = UTF-8 Unicode
% Awesome CV LaTeX Template for CV/Resume
%
% This template has been downloaded from:
% https://github.com/posquit0/Awesome-CV
%
% Author:
% Claud D. Park <posquit0.bj@gmail.com>
% http://www.posquit0.com
%
%
% Adapted to be an Rmarkdown template by Mitchell O'Hara-Wild
% 23 November 2018
%
% Template license:
% CC BY-SA 4.0 (https://creativecommons.org/licenses/by-sa/4.0/)
%
%-------------------------------------------------------------------------------
% CONFIGURATIONS
%-------------------------------------------------------------------------------
% A4 paper size by default, use 'letterpaper' for US letter
\documentclass[11pt, a4paper]{awesome-cv}

% Configure page margins with geometry
\geometry{left=1.4cm, top=.8cm, right=1.4cm, bottom=1.8cm, footskip=.5cm}

% Specify the location of the included fonts
\fontdir[fonts/]

% Color for highlights
% Awesome Colors: awesome-emerald, awesome-skyblue, awesome-red, awesome-pink, awesome-orange
%                 awesome-nephritis, awesome-concrete, awesome-darknight

\colorlet{awesome}{awesome-red}

% Colors for text
% Uncomment if you would like to specify your own color
% \definecolor{darktext}{HTML}{414141}
% \definecolor{text}{HTML}{333333}
% \definecolor{graytext}{HTML}{5D5D5D}
% \definecolor{lighttext}{HTML}{999999}

% Set false if you don't want to highlight section with awesome color
\setbool{acvSectionColorHighlight}{true}

% If you would like to change the social information separator from a pipe (|) to something else
\renewcommand{\acvHeaderSocialSep}{\quad\textbar\quad}

\def\endfirstpage{\newpage}

%-------------------------------------------------------------------------------
%	PERSONAL INFORMATION
%	Comment any of the lines below if they are not required
%-------------------------------------------------------------------------------
% Available options: circle|rectangle,edge/noedge,left/right

\name{JooYoung}{Seo}

\position{RStudio Intern for Summer 2020; Ph.D.~Candidate (ABD)}
\address{Learning, Design, and Technology, 301 Keller Building, University Park, PA 16802, USA}

\mobile{+1 814-777-5825}
\email{\href{mailto:jooyoung@psu.edu}{\nolinkurl{jooyoung@psu.edu}}}
\homepage{jooyoungseo.com}
\github{jooyoungseo}
\linkedin{jooyoungseo}
\twitter{sjystu}

% \gitlab{gitlab-id}
% \stackoverflow{SO-id}{SO-name}
% \skype{skype-id}
% \reddit{reddit-id}

\quote{I am a learning scientist, software engineer, and internationally-certified accessibility professional.}

\usepackage{booktabs}

% Templates for detailed entries
% Arguments: what when with where why
\usepackage{etoolbox}
\def\detaileditem#1#2#3#4#5{%
\cventry{#1}{#3}{#4}{#2}{\ifx#5\empty\else{\begin{cvitems}#5\end{cvitems}}\fi}\ifx#5\empty{\vspace{-4.0mm}}\else\fi}
\def\detailedsection#1{\begin{cventries}#1\end{cventries}}

% Templates for brief entries
% Arguments: what when with
\def\briefitem#1#2#3{\cvhonor{}{#1}{#3}{#2}}
\def\briefsection#1{\begin{cvhonors}#1\end{cvhonors}}

\providecommand{\tightlist}{%
	\setlength{\itemsep}{0pt}\setlength{\parskip}{0pt}}

%------------------------------------------------------------------------------




\begin{document}

% Print the header with above personal informations
% Give optional argument to change alignment(C: center, L: left, R: right)
\makecvheader

% Print the footer with 3 arguments(<left>, <center>, <right>)
% Leave any of these blank if they are not needed
% 2019-02-14 Chris Umphlett - add flexibility to the document name in footer, rather than have it be static Curriculum Vitae
\makecvfooter
  {March 05, 2020}
    {JooYoung Seo~~~·~~~Curriculum Vitae}
  {\thepage}


%-------------------------------------------------------------------------------
%	CV/RESUME CONTENT
%	Each section is imported separately, open each file in turn to modify content
%------------------------------------------------------------------------------



\hypertarget{education}{%
\section{Education}\label{education}}

\detailedsection{\detaileditem{Ph.D. in Learning, Design, and Technology}{Aug. 2016 - Present}{The Pennsylvania State University}{University Park, PA}{\item{Expected graduation: May 2021.}\item{Dissertation Title: ``Discovering Informal Learning Cultures of Blind Individuals Pursuing STEM Disciplines: A Quantitative Ethnography Using Public Listserv Archives.''}\item{Committee members: Drs. Gabriela T. Richard (adviser; dissertation chair), Roy B. Clariana, ChanMin Kim, and Mary Beth Rosson.}}\detaileditem{M.Ed. in Learning, Design, and Technology}{Aug. 2014 - May. 2016}{The Pennsylvania State University}{University Park, PA}{\item{GPA 3.97/4.0.}}\detaileditem{B.A. in Education}{Mar. 2009 - Feb. 2014}{Sungkyunkwan University}{Seoul, South Korea}{\item{GPA 4.08/4.50.}}\detaileditem{B.A. in English Literature}{Mar. 2009 - Feb. 2014}{Sungkyunkwan University}{Seoul, South Korea}{\item{GPA 4.08/4.50.}}}

\hypertarget{working-experience}{%
\section{Working Experience}\label{working-experience}}

\detailedsection{\detaileditem{Summer Intern}{May. 2020 - Aug. 2020}{RStudio}{Redmond, WA}{\empty}\detaileditem{}{}{}{}{\empty}\detaileditem{Co-Founder and Project Manager}{Jul. 2010 - Jun. 2011}{ICE Soft}{Seoul, South Korea}{\empty}\detaileditem{Intern}{May. 2010 - Nov. 2010}{National Human Rights Commission of Korea}{Seoul, South Korea}{\empty}}

\hypertarget{research-experience}{%
\section{Research Experience}\label{research-experience}}

\detailedsection{\detaileditem{Graduate Assistant}{Aug. 2016 - Present}{IT Accessibility Team, Teaching and Learning with Technology (TLT)}{University Park, PA}{\empty}\detaileditem{}{}{}{}{\empty}\detaileditem{Graduate Researcher}{Aug. 2016 - Present}{Playful Learning and Inclusive Design Research Group}{University Park, PA}{\empty}\detaileditem{Principal Investigator}{May. 2019 - Present}{Online Interactions of the Blind Research Project}{University Park, PA}{\empty}\detaileditem{Project Manager}{May. 2018 - Present}{Accessible RMarkdown Online Writer (AROW) Project, Teaching and Learning with Technology (TLT)}{University Park, PA}{\empty}\detaileditem{Principal Investigator}{Jun. 2017 - Present}{Accessibility of Maker Toolkits Research Project}{University Park, PA}{\empty}\detaileditem{Site Manager}{Feb. 2015 - Dec. 2015}{Learning2CC Project}{University Park, PA}{\empty}\detaileditem{Graduate Researcher}{Sep. 2014 - May. 2016}{Avenue PM Research Group}{University Park, PA}{\empty}\detaileditem{Student Researcher}{Feb. 2012 - Mar. 2012}{The Ministry of Knowledge Economy and Seoul National University}{Seoul, South Korea}{\empty}}

\hypertarget{publications}{%
\section{Publications}\label{publications}}

\hypertarget{refereed-journal-papers}{%
\subsection{Refereed Journal Papers}\label{refereed-journal-papers}}

\begingroup
\setlength{\parindent}{-0.5in}
\setlength{\leftskip}{0.5in}

\hypertarget{refs_journals}{}
\leavevmode\hypertarget{ref-seo2019maker}{}%
\textbf{Seo, J.} (2019). Is the maker movement inclusive of ANYONE?: Three accessibility considerations to invite blind makers to the making world. \emph{TechTrends}, \emph{63}(5), 514--520. \url{https://doi.org/10.1007/s11528-019-00377-3}

\leavevmode\hypertarget{ref-seo2019arow}{}%
\textbf{Seo, J.}, \& McCurry, S. (2019). LaTeX is not easy: Creating accessible scientific documents with r markdown. \emph{Journal on Technology and Persons with Disabilities}, \emph{7}, 157--171.

\endgroup

\hypertarget{papers-in-refereed-conference-proceedings}{%
\subsection{Papers in Refereed Conference Proceedings}\label{papers-in-refereed-conference-proceedings}}

\begingroup
\setlength{\parindent}{-0.5in}
\setlength{\leftskip}{0.5in}

\hypertarget{refs_proceedings}{}
\leavevmode\hypertarget{ref-seo2020coding}{}%
\textbf{Seo, J.}, \& Richard, G. T. (in press-). Coding through touch: Exploring and re-designing tactile making activities with learners with visual dis/abilities. In M. Gresalfi \& I. Horn (Eds.), \emph{Interdisciplinarity in the learning sciences, 14th international conference of the learning sciences (icls) 2020}. International Society of the Learning Sciences (ISLS).

\leavevmode\hypertarget{ref-seo2018making}{}%
\textbf{Seo, J.} (2018). Accessibility and inclusivity in making: Engaging learners with all abilities in making activities. In \emph{Proceedings of the 3rd learning sciences graduate student conference} (pp. 141--142). LSGSC Planning Team.

\leavevmode\hypertarget{ref-seo2018accessibility}{}%
\textbf{Seo, J.}, \& Richard, G. T. (2018). Accessibility, making and tactile robotics: Facilitating collaborative learning and computational thinking for learners with visual impairments. In J. Kay \& R. Luckin (Eds.), \emph{Rethinking learning in the digital age: Making the learning sciences count, 13th international conference of the learning sciences (icls) 2018} (Vol. 3, pp. 1755--1757). International Society of the Learning Sciences (ISLS).

\leavevmode\hypertarget{ref-konecki2017role}{}%
Konecki, M., Lovrenčić, S., \textbf{Seo, J.}, \& LaPierre, C. (2017). The role of ict in aiding visually impaired students and professionals. \emph{Proceedings of the 11th Multidisciplinary Academic Conference}, 148.

\leavevmode\hypertarget{ref-seo2017embracing}{}%
\textbf{Seo, J.}, AlQahtani, M., Ouyang, X., \& Borge, M. (2017). Embracing learners with visual impairments in cscl. In B. K. Smith, M. Borge, E. Mercier, \& K. Y. Lim (Eds.), \emph{Making a difference: Prioritizing equity and access in cscl, 12th international conference on computer supported collaborative learning (cscl) 2017} (Vol. 2). International Society of the Learning Sciences (ISLS).

\leavevmode\hypertarget{ref-seo2019discovering}{}%
\textbf{Seo, J.} (2019). Discovering informal learning cultures of blind individuals pursuing stem disciplines: A quantitative ethnography using listserv archives. \emph{The First International Conference on Quantitative Ethnography: Doctoral Consortium}, S66--S67.

\endgroup

\hypertarget{presentations}{%
\section{Presentations}\label{presentations}}

\hypertarget{peer-reviewed-conference-presentations}{%
\subsection{Peer-Reviewed Conference Presentations}\label{peer-reviewed-conference-presentations}}

\begingroup
\setlength{\parindent}{-0.5in}
\setlength{\leftskip}{0.5in}

\textbf{Seo, J.}, \& Richard, G. T. (2020, June). \emph{Coding through touch: exploring and re-designing tactile making activities with learners with visual dis/abilities}. Paper will be presented at the 14th International Conference of the Learning Sciences (ICLS), Nashville, TN.

\textbf{Seo, J.}, \& Richard, G. T. (2020, April). \emph{Maker inclusivity = maker accessibility: further interrogations for diverse participation}. Poster will be presented at the annual meeting of the American Educational Research Association (AERA), San Francisco, CA.

\textbf{Seo, J.} (2019, October). \emph{Discovering informal learning cultures of blind individuals pursuing STEM disciplines: a quantitative ethnography using listserv archives}. Poster presented at the Doctoral Consortium of the first International Conference on Quantitative Ethnography (ICQE), Madison, WI. \emph{Awarded the best doctoral proposal fellowship}.

\textbf{Seo, J.}, \& McCurry, S. (2019, March). \emph{LaTeX is NOT easy: creating accessible scientific documents with R markdown}. Paper presented at the annual meeting of the CSUN Assistive Technology Conference, Northridge, CA.

\endgroup

\end{document}
