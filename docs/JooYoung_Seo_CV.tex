%!TEX TS-program = xelatex
%!TEX encoding = UTF-8 Unicode
% Awesome CV LaTeX Template for CV/Resume
%
% This template has been downloaded from:
% https://github.com/posquit0/Awesome-CV
%
% Author:
% Claud D. Park <posquit0.bj@gmail.com>
% http://www.posquit0.com
%
%
% Adapted to be an Rmarkdown template by Mitchell O'Hara-Wild
% 23 November 2018
%
% Template license:
% CC BY-SA 4.0 (https://creativecommons.org/licenses/by-sa/4.0/)
%
%-------------------------------------------------------------------------------
% CONFIGURATIONS
%-------------------------------------------------------------------------------
% A4 paper size by default, use 'letterpaper' for US letter
\documentclass[11pt,a4paper,]{awesome-cv}

% Configure page margins with geometry
\usepackage{geometry}
\geometry{left=1.4cm, top=.8cm, right=1.4cm, bottom=1.8cm, footskip=.5cm}


% Specify the location of the included fonts
\fontdir[fonts/]

% Color for highlights
% Awesome Colors: awesome-emerald, awesome-skyblue, awesome-red, awesome-pink, awesome-orange
%                 awesome-nephritis, awesome-concrete, awesome-darknight

\definecolor{awesome}{HTML}{414141}

% Colors for text
% Uncomment if you would like to specify your own color
% \definecolor{darktext}{HTML}{414141}
% \definecolor{text}{HTML}{333333}
% \definecolor{graytext}{HTML}{5D5D5D}
% \definecolor{lighttext}{HTML}{999999}

% Set false if you don't want to highlight section with awesome color
\setbool{acvSectionColorHighlight}{true}

% If you would like to change the social information separator from a pipe (|) to something else
\renewcommand{\acvHeaderSocialSep}{\quad\textbar\quad}

\def\endfirstpage{\newpage}

%-------------------------------------------------------------------------------
%	PERSONAL INFORMATION
%	Comment any of the lines below if they are not required
%-------------------------------------------------------------------------------
% Available options: circle|rectangle,edge/noedge,left/right

\name{JooYoung}{Seo}

\position{Ph.D.}
\address{614 E. Daniel St.~\textbar{} Room 5158, Champaign, IL 61820}

\pronouns{he/him/his}
\mobile{+1 217-333-2671}
\email{\href{mailto:jseo1005@illinois.edu}{\nolinkurl{jseo1005@illinois.edu}}}
\homepage{jooyoungseo.github.io}
\github{jooyoungseo}
\linkedin{jooyoungseo}
\twitter{seo\_jooyoung}

% \gitlab{gitlab-id}
% \stackoverflow{SO-id}{SO-name}
% \skype{skype-id}
% \reddit{reddit-id}

\quote{I am a learning scientist, data-science/software developer, and
internationally certified accessibility professional.}

\usepackage{booktabs}

\providecommand{\tightlist}{%
	\setlength{\itemsep}{0pt}\setlength{\parskip}{0pt}}

%------------------------------------------------------------------------------


\usepackage{xurl}

% Pandoc CSL macros
\newlength{\cslhangindent}
\setlength{\cslhangindent}{1.5em}
\newlength{\csllabelwidth}
\setlength{\csllabelwidth}{2em}
\newenvironment{CSLReferences}[2] % #1 hanging-ident, #2 entry spacing
 {% don't indent paragraphs
  \setlength{\parindent}{0pt}
  % turn on hanging indent if param 1 is 1
  \ifodd #1 \everypar{\setlength{\hangindent}{\cslhangindent}}\ignorespaces\fi
  % set entry spacing
  \ifnum #2 > 0
  \setlength{\parskip}{#2\baselineskip}
  \fi
 }%
 {}
\usepackage{calc}
\newcommand{\CSLBlock}[1]{#1\hfill\break}
\newcommand{\CSLLeftMargin}[1]{\parbox[t]{\csllabelwidth}{\honortitlestyle{#1}}}
\newcommand{\CSLRightInline}[1]{\parbox[t]{\linewidth - \csllabelwidth}{\honordatestyle{#1}}}
\newcommand{\CSLIndent}[1]{\hspace{\cslhangindent}#1}

\begin{document}

% Print the header with above personal informations
% Give optional argument to change alignment(C: center, L: left, R: right)
\makecvheader

% Print the footer with 3 arguments(<left>, <center>, <right>)
% Leave any of these blank if they are not needed
% 2019-02-14 Chris Umphlett - add flexibility to the document name in footer, rather than have it be static Curriculum Vitae
\makecvfooter
  {August 31, 2023}
    {JooYoung Seo~~~·~~~Curriculum Vitae}
  {\thepage~ of \pageref{LastPage}~}


%-------------------------------------------------------------------------------
%	CV/RESUME CONTENT
%	Each section is imported separately, open each file in turn to modify content
%------------------------------------------------------------------------------



\section{Education}\label{education}

\begin{cventries}
    \cventry{Ph.D. in Learning, Design, and Technology}{The Pennsylvania State University}{University Park, PA}{2021}{\begin{cvitems}
\item Dissertation Title: ``Discovering Informal Learning Cultures of Blind Individuals Pursuing STEM Disciplines: A Computational Ethnography Using Public Listserv Archives.''
\item Committee members: Drs. Gabriela T. Richard (adviser; dissertation chair), Roy B. Clariana, ChanMin Kim, and Mary Beth Rosson.
\end{cvitems}}
    \cventry{M.Ed. in Learning, Design, and Technology}{The Pennsylvania State University}{University Park, PA}{2016}{}\vspace{-4.0mm}
    \cventry{Double B.A. in Education, English Literature}{Sungkyunkwan University}{Seoul, South Korea}{2014}{}\vspace{-4.0mm}
\end{cventries}

\section{Professional Appointment}\label{professional-appointment}

\begin{cventries}
    \cventry{Assistant Professor}{School of Information Sciences}{University of Illinois at Urbana-Champaign}{Sep. 2021 - Present}{}\vspace{-4.0mm}
    \cventry{Faculty Affiliate}{National Center for Supercomputing Applications}{University of Illinois at Urbana-Champaign}{Dec. 2021 - Present}{}\vspace{-4.0mm}
    \cventry{Faculty Affiliate}{Illinois Informatics Institute}{University of Illinois at Urbana-Champaign}{Sep. 2021 - Present}{}\vspace{-4.0mm}
    \cventry{Faculty Affiliate}{IDEA (Inclusion, Diversity, Equity, and Access) Institute}{University of Illinois at Urbana-Champaign}{Jan. 2022 - Present}{}\vspace{-4.0mm}
    \cventry{Affiliate Scientist}{Rehabilitation Engineering Research Center (RERC) on Blindness and Low Vision}{Smith-Kettlewell Eye Research Institute}{Jul. 2023 - Present}{}\vspace{-4.0mm}
    \cventry{Software Engineer Intern}{RStudio}{Boston, MA}{May. 2020 - Aug. 2020}{}\vspace{-4.0mm}
\end{cventries}

\section{Grants}\label{grants}

\begin{itemize}
\item
  ``Beyond Visuals: Improving Accessibility of Data Curation and
  Multi-Modal Representations for People of all Abilities through
  Reproducible Workflows,'' PI, Institute of Museum and Library Services
  (IMLS) Laura Bush 21st Century Librarian Program, Grant
  \#RE-254891-OLS-23, \$649,921, Aug 2023 - July 2026 (ongoing).
\item
  ``Promoting Computational Thinking Skills for Blind and Visually
  Impaired Teens through Accessible Library Makerspaces,'' PI, Institute
  of Museum and Library Services (IMLS) National Leadership Grants,
  Grant \#LG-252360-OLS-22, \$498,638, Aug 2022 - May 2025 (ongoing).
\item
  ``Inclusive Data Science: Fostering Accessibility and Reproducibility
  with Multimodal Learning Strategies,'' PI, Teach Access Faculty
  Grants, Tier 3, \$5,000, May 2023 - May 2024 (ongoing).
\item
  ``Improving Accessible Reproducibility for Data Science Publishing
  System,'' PI, Posit PBC (unrestricted gift), \$54,327, May 2022 -
  Present (ongoing).
\item
  ``Co-designing an Alexa-Based Conversational mHealth System with
  Visually Impaired People to Promote Physical Activities,'' Co-PI,
  Center on Health, Aging, and Disability's (CHAD) Pilot Grant Program
  at the University of Illinois at Urbana-Champaign, \$29,866, 2023-2024
  (ongoing).
\item
  ``Category I: Crossing the Divide Between Today's Practice and
  Tomorrow's Science,'' Senior Personnel, National Science Foundation
  (NSF), Grant \#2005572, \$10,000,000.00, Oct 2021 - Sept 30, 2026 (.5
  FTE academic year) (ongoing).
\item
  ``Insight without Sight: Enhancing Data Literacy through Multimodal
  Data Science Education - Smith-Kettlewell Summer Institute,'' Subaward
  PI, Rehabilitation Engineering Research Centers (RERC) Program: RERC
  on Blindness and Low Vision, National Institute on Disability,
  Independent Living, and Rehabilitation Research (NIDILRR), Grant
  \#90REGE0018-01-00, \$11,975, July 17, 2023 - August 11, 2023
  (completed).
\item
  ``Data Accessibilization: Making Data Science Education Accessible for
  Blind Learners,'' PI, Wallace Foundation, Emerging Scholars Program,
  International Society of the Learning Sciences, \$10,000, Jan 2022 -
  Dec 2022 (completed).
\end{itemize}

\section{Publications}\label{publications}

\begin{itemize}
\tightlist
\item
  The following notations only apply to September 2021 and after.
\end{itemize}

\texttt{\#}: derived from the candidate's thesis.

\texttt{*}: publication that has undergone stringent editorial review by
peers.

\texttt{+}: publication that was invited and carries special prestige
and recognition.

\subsection{Refereed Journal Papers}\label{refereed-journal-papers}

\hypertarget{bibliography}{}
\leavevmode\vadjust pre{\hypertarget{ref-moon2023revisiting}{}}%
1.\textsuperscript{\textbf{\emph{*}}} Moon, J., Choi, G. W., \&
\textbf{Seo, J.} (accepted). Revisiting multimedia learning design
principles in virtual reality-based learning environments for autistic
individuals. \emph{Virtual Reality}.

\leavevmode\vadjust pre{\hypertarget{ref-leePersonalHealthData2023}{}}%
2.\textsuperscript{\textbf{\emph{*}}} Lee, J. G. W., Lee, K., Lee, B.,
Choi, S., \textbf{Seo, J.}, \& Choe, E. K. (2023). Personal Health Data
Tracking by Blind and Low-Vision People: Survey Study. \emph{Journal of
Medical Internet Research}, \emph{25}(1), e43917.
\url{https://doi.org/10.2196/43917}

\leavevmode\vadjust pre{\hypertarget{ref-moonLearningAnalyticsSeamless2023a}{}}%
3.\textsuperscript{\textbf{\emph{*}}} Moon, J., Lee, D., Choi, G. W.,
\textbf{Seo, J.}, Do, J., \& Lim, T. (2023). Learning analytics in
seamless learning environments: A systematic review. \emph{Interactive
Learning Environments}, \emph{0}(0), 1--18.
\url{https://doi.org/10.1080/10494820.2023.2170422}

\leavevmode\vadjust pre{\hypertarget{ref-seoTeachingVisualAccessibility2023}{}}%
4.\textsuperscript{\textbf{\emph{*}}} \textbf{Seo, J.}, \& Dogucu, M.
(2023). Teaching Visual Accessibility in Introductory Data Science
Classes with Multi-Modal Data Representations. \emph{Journal of Data
Science}, 1--14. \url{https://doi.org/10.6339/23-JDS1095}

\leavevmode\vadjust pre{\hypertarget{ref-seo2022csun}{}}%
5.\textsuperscript{\textbf{\emph{*}}} \textbf{Seo, J.}, \& Choi, S.
(2022). Are blind people considered a part of scientific knowledge
producers?: Accessibility report on top-10 SCIE journal systems using a
tripartite evaluation approach. \emph{Journal on Technology and Persons
with Disabilities}.

\leavevmode\vadjust pre{\hypertarget{ref-seoScopingReviewThree2022}{}}%
6.\textsuperscript{\textbf{\emph{*}}} \textbf{Seo, J.}, Moon, J., Choi,
G. W., \& Do, J. (2022). A Scoping Review of Three Computational
Approaches to Ethnographic Research in Digital Learning Environments.
\emph{TechTrends}, \emph{66}(1), 102--111.
\url{https://doi.org/10.1007/s11528-021-00689-3}

\leavevmode\vadjust pre{\hypertarget{ref-doi:10.1080ux2f09286586.2020.1863993}{}}%
7. Choi, S., \& \textbf{Seo, J.} (2021). Trends in healthcare research
on visual impairment and blindness: Use of bibliometrics and
hierarchical cluster analysis. \emph{Ophthalmic Epidemiology},
\emph{28}(4), 277--284.
\url{https://doi.org/10.1080/09286586.2020.1863993}

\leavevmode\vadjust pre{\hypertarget{ref-seoSCAFFOLDingAllAbilities2021a}{}}%
8.\textsuperscript{\textbf{\emph{*}}} \textbf{Seo, J.}, \& Richard, G.
T. (2021). SCAFFOLDing All Abilities into Makerspaces: A Design
Framework for Universal, Accessible and Intersectionally Inclusive
Making and Learning. \emph{Information and Learning Sciences},
\emph{122}(11/12), 795--815.
\url{https://doi.org/10.1108/ILS-10-2020-0230}

\leavevmode\vadjust pre{\hypertarget{ref-seo2019maker}{}}%
9. \textbf{Seo, J.} (2019). Is the maker movement inclusive of ANYONE?:
Three accessibility considerations to invite blind makers to the making
world. \emph{TechTrends}, \emph{63}(5), 514--520.
\url{https://doi.org/10.1007/s11528-019-00377-3}

\leavevmode\vadjust pre{\hypertarget{ref-seo2019arow}{}}%
10. \textbf{Seo, J.}, \& McCurry, S. (2019). LaTeX is NOT easy: Creating
accessible scientific documents with r markdown. \emph{Journal on
Technology and Persons with Disabilities}, \emph{7}, 157--171.

\subsection{Papers in Refereed Conference
Proceedings}\label{papers-in-refereed-conference-proceedings}

\hypertarget{bibliography}{}
\leavevmode\vadjust pre{\hypertarget{ref-kim2023trend}{}}%
1.\textsuperscript{\textbf{\emph{*}}} Kim, S. H., Yoon, A., \&
\textbf{Seo, J.} (2023). Trend of collaboration in STEM education in
informal learning institutions based on IMLS-funded projects.
\emph{Annual Meeting 2023 Association for Information Science and
Technology}. Forthcoming.

\leavevmode\vadjust pre{\hypertarget{ref-park2023datadriven}{}}%
2.\textsuperscript{\textbf{\emph{*}}} Park, J., \textbf{Seo, J.}, \&
Lee, J. Y. (2023). A data-driven approach towards understanding
collaborative problem-solving of blind programmers in online community.
\emph{Annual Meeting 2023 Association for Information Science and
Technology}. Forthcoming.

\leavevmode\vadjust pre{\hypertarget{ref-seo2023coding}{}}%
3.\textsuperscript{\textbf{\emph{*}}} \textbf{Seo, J.}, \& Rogge, M.
(2023). Coding non-visually in visual studio code: Collaboration towards
accessible development environment for blind programmers. \emph{The 25th
International ACM SIGACCESS Conference on Computers and Accessibility}.
Forthcoming.

\leavevmode\vadjust pre{\hypertarget{ref-289508}{}}%
4.\textsuperscript{\textbf{\emph{*}}} Kaushik, S., Barbosa, N. M., Yu,
Y., Sharma, T., Kilhoffer, Z., \textbf{Seo, J.}, Das, S., \& Wang, Y.
(2023). GuardLens: Supporting safer online browsing for people with
visual impairments. \emph{Nineteenth Symposium on Usable Privacy and
Security (SOUPS 2023)}, 361--380.
\url{https://www.usenix.org/conference/soups2023/presentation/kaushik}

\leavevmode\vadjust pre{\hypertarget{ref-seoMAIDRMultimodalAccessInPress}{}}%
5.\textsuperscript{\textbf{\emph{*}}} \textbf{Seo, J.}, Xia, Y., Yam, Y.
J., \& McCurry, S. (2023). \emph{MAIDR: Multimodal Access and
Interactive Data Representation System for Inclusive Data Science
Education}. 2023 International Conference of the Learning Sciences.

\leavevmode\vadjust pre{\hypertarget{ref-289496}{}}%
6.\textsuperscript{\textbf{\emph{*}}} Zhang, Z. (Jerry)., Kaushik, S.,
\textbf{Seo, J.}, Yuan, H., Das, S., Findlater, L., Gurari, D., Stangl,
A., \& Wang, Y. (2023). ImageAlly: A Human-AI hybrid approach to support
blind people in detecting and redacting private image content.
\emph{Nineteenth Symposium on Usable Privacy and Security (SOUPS 2023)},
417--436.
\url{https://www.usenix.org/conference/soups2023/presentation/zhang}

\leavevmode\vadjust pre{\hypertarget{ref-seo2022cscl}{}}%
7.\textsuperscript{\textbf{\emph{*}}} Huh, M., \& \textbf{Seo, J.}
(2022). A duoethnographic study of a mixed-ability team in a
collaborative group programming project. In \emph{Proceedings of the
15th international conference on computer supported collaborative
learning (CSCL), 2022}. International Society of the Learning Sciences
(ISLS).

\leavevmode\vadjust pre{\hypertarget{ref-leeCollabAllyAccessibleCollaboration2022a}{}}%
8.\textsuperscript{\textbf{\emph{*}}} Lee, C. Y. P., Zhang, Z.,
Herskovitz, J., \textbf{Seo, J.}, \& Guo, A. (2022). CollabAlly:
Accessible Collaboration Awareness in Document Editing. \emph{CHI
Conference on Human Factors in Computing Systems}, 1--17.
\url{https://doi.org/10.1145/3491102.3517635}. \emph{Honorable mention
award}.

\leavevmode\vadjust pre{\hypertarget{ref-teachingvisualaccessibility2022}{}}%
9.\textsuperscript{\textbf{\emph{*}}} \textbf{Seo, J.}, \& Dogucu, M.
(2022). Teaching visual accessibility in the introductory data science
classes: Why, what, when, and how. \emph{The 2022 Symposium on Data
Science \& Statistics}.

\leavevmode\vadjust pre{\hypertarget{ref-lee2021collabally}{}}%
10.\textsuperscript{\textbf{\emph{*}}} Lee, C. Y. P., Zhang, Z.,
Herskovitz, J., \textbf{Seo, J.}, \& Guo, A. (2021). CollabAlly:
Accessible collaboration awareness in document editing. \emph{The 23rd
International ACM SIGACCESS Conference on Computers and Accessibility},
1--4. \url{https://doi.org/10.1145/3441852.3476562}

\leavevmode\vadjust pre{\hypertarget{ref-seo2020coding}{}}%
11. \textbf{Seo, J.}, \& Richard, G. T. (2020). Coding through touch:
Exploring and re-designing tactile making activities with learners with
visual dis/abilities. In M. Gresalfi \& I. Horn (Eds.),
\emph{Interdisciplinarity in the learning sciences, 14th international
conference of the learning sciences (ICLS) 2020} (Vol. 3, pp.
1373--1380). International Society of the Learning Sciences (ISLS).

\leavevmode\vadjust pre{\hypertarget{ref-seo2019discovering}{}}%
12. \textbf{Seo, J.} (2019). Discovering informal learning cultures of
blind individuals pursuing STEM disciplines: A quantitative ethnography
using listserv archives. \emph{The First International Conference on
Quantitative Ethnography: Doctoral Consortium}, S66--S67. \emph{Awarded
the best Doctoral Consortium Proposal Cengage fellowship}.

\leavevmode\vadjust pre{\hypertarget{ref-seo2018making}{}}%
13. \textbf{Seo, J.} (2018). Accessibility and inclusivity in making:
Engaging learners with all abilities in making activities. In
\emph{Proceedings of the 3rd learning sciences graduate student
conference} (pp. 141--142). LSGSC Planning Team.

\leavevmode\vadjust pre{\hypertarget{ref-seo2018accessibility}{}}%
14. \textbf{Seo, J.}, \& Richard, G. T. (2018). Accessibility, making
and tactile robotics: Facilitating collaborative learning and
computational thinking for learners with visual impairments. In J. Kay
\& R. Luckin (Eds.), \emph{Rethinking learning in the digital age:
Making the learning sciences count, 13th international conference of the
learning sciences (ICLS) 2018} (Vol. 3, pp. 1755--1757). International
Society of the Learning Sciences (ISLS).

\leavevmode\vadjust pre{\hypertarget{ref-konecki2017role}{}}%
15. Konecki, M., Lovrenčić, S., \textbf{Seo, J.}, \& LaPierre, C.
(2017). The role of ICT in aiding visually impaired students and
professionals. \emph{Proceedings of the 11th Multidisciplinary Academic
Conference}, 148.

\leavevmode\vadjust pre{\hypertarget{ref-seo2017embracing}{}}%
16. \textbf{Seo, J.}, AlQahtani, M., Ouyang, X., \& Borge, M. (2017).
Embracing learners with visual impairments in CSCL. In B. K. Smith, M.
Borge, E. Mercier, \& K. Y. Lim (Eds.), \emph{Making a difference:
Prioritizing equity and access in CSCL, 12th international conference on
computer supported collaborative learning (CSCL) 2017} (Vol. 2, pp.
573--576). International Society of the Learning Sciences (ISLS).

\subsection{Publications in
Healthcare}\label{publications-in-healthcare}

\hypertarget{bibliography}{}
\leavevmode\vadjust pre{\hypertarget{ref-doi:10.1080ux2f01612840.2019.1705944}{}}%
1. Choi, S., \& \textbf{Seo, J.} (2020). An exploratory study of the
research on caregiver depression: Using bibliometrics and LDA topic
modeling. \emph{Issues in Mental Health Nursing}, 1--10.
\url{https://doi.org/10.1080/01612840.2019.1705944}

\leavevmode\vadjust pre{\hypertarget{ref-doi:10.1111ux2fnuf.12328}{}}%
2. Choi, S., \& \textbf{Seo, J.} (2019). Analysis of caregiver burden in
palliative care: An integrated review. \emph{Nursing Forum},
\emph{54}(2), 280--290. \url{https://doi.org/10.1111/nuf.12328}

\leavevmode\vadjust pre{\hypertarget{ref-choi2019heart}{}}%
3. Choi, S., \& \textbf{Seo, J.} (2019). Heart failure research using
text mining: A systematic review. \emph{NURSING RESEARCH}, \emph{68}(2),
E119--E120.

\leavevmode\vadjust pre{\hypertarget{ref-choi2019trends}{}}%
4. Choi, S., \& \textbf{Seo, J.} (2019). Trends in self-management among
adults with heart failure from 2011 to 2016 using the national health
and nutrition examination surveys. \emph{Journal of Cardiac Failure},
\emph{25}(8), S4. \url{https://doi.org/10.1016/j.cardfail.2019.07.540}

\leavevmode\vadjust pre{\hypertarget{ref-choi2019exploring}{}}%
5. Choi, S., \textbf{Seo, J.}, \& Kitko, L. (2019). Exploring the lived
experience of a family member with advanced heart failure: Using a text
mining approach. \emph{The First International Conference on
Quantitative Ethnography: Poster Session}, S8--S9.

\leavevmode\vadjust pre{\hypertarget{ref-choi2018effects}{}}%
6. Choi, S., \& \textbf{Seo, J.} (2018). Effects of nonpharmacological
interventions for fatigue in patients with heart failure: A systematic
review and meta-analysis. \emph{Journal of Cardiac Failure},
\emph{24}(8), S73--S74.

\section{Conference Presentations and Invited
Talks}\label{conference-presentations-and-invited-talks}

\textbf{Seo, J.} (2021, October). \emph{Non-Visual Strategies for Making
Statistical-Computing Accessible}. Lightening talk at the National
Center for Supercomputing Applications, Champaign, IL.

\textbf{Seo, J.} (2021, September). \emph{Accessibility of Science
Beyond Content Accessibility}. Talk presented at the Argonne National
Laboratory (Maths and Computer Science Division):
\emph{\url{https://www.anl.gov/event/accessibility-of-science-beyond-content-accessibility}}.

Invited panel (2021, August). DO-IT Neuroscience for Neurodiverse
Learners Faculty panel at the University of Washington, virtual.

\textbf{Seo, J.} (2021, June). \emph{How to Learn to Code}. Invited Talk
presented at the Nature, Webcast:
\emph{\url{https://doi.org/10.1038/d41586-021-01638-z}}.

\textbf{Seo, J.}, \& Richard, G. T. (2021, February). \emph{Uncovering
latent topics of blind people in computer science: structural topic
modeling for an email corpus}. Poster presented at the 2nd International
Conference on Quantitative Ethnography (ICQE), Malibu, CA.

Donegan, S. R., Porter, C., Fogel, A., \textbf{Seo, J.}, Choi, S., \&
Eagan, B. (2021, February). \emph{U.S. media coverage during COVID-19:
an epistemic network analysis of bias, topic, and trajectory}. Poster
presented at the 2nd International Conference on Quantitative
Ethnography (ICQE), Malibu, CA.

\textbf{Seo, J.} (2021, January). \emph{Accessible data science beyond
visual models}. Talk presented at the rstudio::global(2021), Virtual:
\emph{\url{https://global.rstudio.com/student/page/40617}}.

\textbf{Seo, J.} (2020, September). \emph{Discovering knowledge sharing
patterns of blind people pursuing STEM disciplines: data science and
computational linguistics on large-scale email corpora}. Poster
presented at the Doctoral Consortium of the annual meeting of the ACM
Richard Tapia Celebration of Diversity in Computing, virtual.
\emph{Awarded the Qualcomm scholarship}.

\textbf{Seo, J.}, \& Richard, G. T. (2020, April). \emph{Maker
inclusivity = maker accessibility: further interrogations for diverse
participation}. Poster presented at the annual meeting of the American
Educational Research Association (AERA), Virtual.

\textbf{Seo, J.}, \& Richard, G. T. (2018, April). \emph{Furthering
inclusivity in making: a framework for accessible design of makerspaces
for learners with disabilities}. Poster presented at the annual meeting
of the American Educational Research Association (AERA), New York City,
NY.

Bunag, T., Aniela, L., Nielsen, M. C., \& \textbf{Seo, J.} (2017,
November). \emph{The rapidly changing world of accessible online
learning}. Presented at the Panel Discussion: DDL - Accessible Online
Learning In Concurrent Presentation of the Association for Educational
Communications and Technology (AECT), Jacsonville, FL.

\textbf{Seo, J.} (2017, September). \emph{Tactile access to visualized
statistical data using R}. Poster presented at the annual meeting of the
ACM Richard Tapia Celebration of Diversity in Computing, Atlanta, GA.

Liao, J., Patcyk, M., \textbf{Seo, J.}, \& Hooper, S. (2016, October).
\emph{Using hierarchical linear modeling to measure growth rate in a
gamified CBM environment}. Paper presented at the annual meeting of the
Northeastern Educational Research Association (NERA), Trumbull, CT.

\textbf{Seo, J.} (2016, March). \emph{Engaging blind learners in
statistics study using R}. Presented at the annual meeting of the
Teaching and Learning with Technology (TLT) Symposium, University Park,
PA.

Kim, K., \textbf{Seo, J.}, \& Clariana, R. B. (2016, March).
\emph{Automatic knowledge structure measure in online courses}.
Presented at the annual meeting of the Teaching and Learning with
Technology (TLT) Symposium, University Park, PA.

\textbf{Seo, J.} (2015, November). \emph{Assistive technologies for
equal access in general education}. Presented at the annual meeting of
the Association for Educational Communications and Technology (AECT),
Indianapolis, IN.

\textbf{Seo, J.}, \& Park, E. (2015, October). \emph{The more
accessible, the more potential: simple tips for online accessibility}.
Presented at the Technology and Learning Conference, Blue Bell, PA.

\section{Software Developments and
Publications}\label{software-developments-and-publications}

\subsection{Data Science Packages in Comprehensive R Archive Network
(CRAN)}\label{data-science-packages-in-comprehensive-r-archive-network-cran}

\hypertarget{bibliography}{}
\leavevmode\vadjust pre{\hypertarget{ref-R-youtubecaption}{}}%
1. \textbf{Seo, J.}, \& Choi, S. (2020). \emph{Youtubecaption:
Downloading YouTube subtitle transcription in a tidy tibble data frame}.
\url{https://CRAN.R-project.org/package=youtubecaption}. \emph{Over 9543
download}.

\leavevmode\vadjust pre{\hypertarget{ref-R-ezpickr}{}}%
2. \textbf{Seo, J.}, \& Choi, S. (2019). \emph{Ezpickr: Easy data import
using GUI file picker and seamless communication between an excel and
r}. \url{https://CRAN.R-project.org/package=ezpickr}. \emph{Over 19K
download}.

\leavevmode\vadjust pre{\hypertarget{ref-R-mboxr}{}}%
3. \textbf{Seo, J.}, \& Choi, S. (2019). \emph{Mboxr: Reading,
extracting, and converting an mbox file into a tibble}.
\url{https://CRAN.R-project.org/package=mboxr}. \emph{Over 15K
download}.

\newpage

\subsection{Open-Source Project on
GitHub}\label{open-source-project-on-github}

\hypertarget{bibliography}{}
\leavevmode\vadjust pre{\hypertarget{ref-R-edmdown}{}}%
1. \textbf{Seo, J.} (2020). \emph{Edmdown: Writing a reproducible
article for journal of educational DataMining in r markdown}.
\url{https://github.com/jooyoungseo/edmdown}. \emph{R package version
0.0.0.9000}.

\leavevmode\vadjust pre{\hypertarget{ref-R-islsdown}{}}%
2. \textbf{Seo, J.}, Chan, T., Michaelis, J. E., \& Rosenberg, J. M.
(2020). \emph{Islsdown: Writing a reproducible conference paper for the
international society of the learning sciences annual meeting in r
markdown}. \url{https://github.com/jooyoungseo/islsdown}. \emph{R
package version 0.0.0.9000}.

\leavevmode\vadjust pre{\hypertarget{ref-R-jladown}{}}%
3. \textbf{Seo, J.}, \& Rosenberg, J. M. (2020). \emph{Jladown: Writing
a reproducible article for journal of learning analytics in r markdown}.
\url{https://github.com/jooyoungseo/jladown}. \emph{R package version
0.0.0.9000}.

\leavevmode\vadjust pre{\hypertarget{ref-R-ezviewr}{}}%
4. \textbf{Seo, J.} (2019). \emph{Ezviewr: View tidy data in preferable
spreadsheet}. \url{https://github.com/jooyoungseo/ezviewr}. \emph{R
package version 0.1.0}.

\leavevmode\vadjust pre{\hypertarget{ref-R-tactileR}{}}%
5. \textbf{Seo, J.} (2019). \emph{tactileR: Converting r graphics into a
braille ready-to-print PDF}.
\url{https://github.com/jooyoungseo/tactileR}. \emph{R package version
0.1.0}.

\leavevmode\vadjust pre{\hypertarget{ref-webrender}{}}%
6. \textbf{Seo, J.}, \& McCurry, S. (2019). \emph{AROW: Accessible
RMarkdown online writer}. \url{http://www.arowtool.com}

\subsection{Officially Code\_Contributing R
Packages}\label{officially-code_contributing-r-packages}

\begin{itemize}
\tightlist
\item
  \href{https://github.com/pulls?q=is%3Apr+author%3Ajooyoungseo+archived%3Afalse+is%3Aclosed}{My
  GitHub pull requests} have been peer-reviewed and merged for active
  data science R packages:
\end{itemize}

\begin{cventries}
    \cventry{shiny: Web Application Framework for R}{Chang, W., Cheng, J., Allaire, J., Xie, Y., \& McPherson, J.}{https://cran.r-project.org/web/packages/shiny/}{2020}{\begin{cvitems}
\item Contributed to improving keyboard and TTS accessibility for input widgets.
\end{cvitems}}
    \cventry{vitae: Curriculum Vitae for R Markdown}{O'Hara-Wild, M., \& Hyndman, R.}{https://cran.r-project.org/web/packages/vitae/}{2020}{\begin{cvitems}
\item Contributed to enabling LaTeX engine customization.
\item Contributed to displaying multiple bibliographies based on lua filter.
\end{cvitems}}
    \cventry{rmarkdown: Dynamic Documents for R}{Allaire, J., Xie, Y., et al.}{https://cran.r-project.org/web/packages/rmarkdown/}{2019}{\begin{cvitems}
\item Contributed to improving ioslides\_presentation accessibility for screen reading software.
\end{cvitems}}
    \cventry{bookdown: Authoring Books and Technical Documents with R Markdown}{Xie, Y.}{https://cran.r-project.org/web/packages/bookdown/}{2019}{\begin{cvitems}
\item Developed context\_document2, github\_document2, beamer\_presentation2, html\_fragment2, html\_notebook2, html\_vignette2, ioslides\_presentation2, slidy\_presentation2 and rtf\_document2 functions to enable R markdown users to utilize cross-references for figures and tables through bookdown package.
\end{cvitems}}
    \cventry{distill: 'R Markdown' Format for Scientific and Technical Writing}{Allaire, J., Iannone, R., \& Xie, Y.}{https://cran.r-project.org/web/packages/distill/}{2019}{\begin{cvitems}
\item Contributed to displaying more metadata for journal article bib\_entries.
\item Improved screen reader accessibility for navigation bar.
\item Added missing Google Scholar meta tags for conference, technical report, and dissertation types.
\end{cvitems}}
    \cventry{wordcountaddin: Word counts and readability statistics in R markdown documents}{Marwick, B.}{https://git.io/JkJca}{2019}{\begin{cvitems}
\item Contributed to text\_stats() function for R markdown users to Get a word count and some other stats for selected text (excluding code chunks and inline code).
\end{cvitems}}
    \cventry{BrailleR: Improved Access for Blind Users}{Godfrey, A. J. R., Warren, D., Murrell, P., Bilton, T., \& Sorge, V.}{https://cran.r-project.org/web/packages/BrailleR/}{2019}{\begin{cvitems}
\item Contributed to BRLThis() function to enable blind R users to convert a graph into a pdf ready for embossing in Braille.
\end{cvitems}}
\end{cventries}

\section{Teaching Experience}\label{teaching-experience}

\begin{itemize}
\item
  IS 407: Introduction to Data Science (spring 2022).
\item
  IS 569: Internship (spring 2022). \emph{Advised Ankita Khiratkar (MS
  IM)}.
\item
  IS 589: Independent Study (spring 2022). \emph{Advised Chetna Beri (MS
  IM)}.
\item
  INFO 399: Individual Study (spring 2022). \emph{Advised Leo Zhang (CS
  undergrad --\textgreater{} Currently, Master's student at UCLA Data
  Science dept)}.
\item
  IS 389: Independent Study (spring 2022). \emph{Advised Miso Kim (BS
  IS)}.
\item
  IS 226: Introduction to Human-Computer Interaction (fall 2021).
  \emph{Co-teaching with Dr.~Yang Wang}.
\item
  Served as an annual review committee member for iSchool Ph.d. student
  Smirity Kaushik (advisor Dr.~Yang Wang) (spring 2022).
\item
  Served as a field exam committee member for iSchool Ph.D.~candidate
  Jaihyun Park (advisor Dr.~Ryan Cordell) (spring 2022).
\end{itemize}

\section{Awards and Honors}\label{awards-and-honors}

\begin{cventries}
    \cventry{Qualcomm Scholarship}{ACM Richard Tapia Celebration of Diversity in Computing Conference}{Virtual}{Sep. 2020}{\begin{cvitems}
\item Selected as one of the outstanding doctoral consortium proposals.
\end{cvitems}}
    \cventry{2020 NextGen Leaders}{Disability:IN}{Alexandria, VA}{Feb. 2020 - Jul. 2020}{\begin{cvitems}
\item Selected as one of the leaders in the U.S. for diversity.
\end{cvitems}}
    \cventry{AccessSTEM Resume contest winners Award}{AccessComputing, University of Washington}{Seattle, WA}{Dec. 2019}{\begin{cvitems}
\item Sponsored by The National Science Foundation (NSF).
\end{cvitems}}
    \cventry{Best Doctoral Proposal Fellowship}{The First International Conference on Quantitative Ethnography}{Madison, WI}{Oct. 2019}{\begin{cvitems}
\item Sponsored by Cengage.
\end{cvitems}}
    \cventry{Andrew V. Kozak Memorial Fellowship}{The PDK Educational Foundation}{University Park, PA}{2019}{\begin{cvitems}
\item Awarded \$1,500 for research contribution to public education.
\end{cvitems}}
    \cventry{AERA Pre-Conference Travel Award}{The Spencer Foundation}{Toronto, ON, Canada}{Apr. 2019}{}\vspace{-4.0mm}
    \cventry{Delta Gamma Golden Anchor Award}{The Pennsylvania State University}{University Park, PA}{2015, 2017}{}\vspace{-4.0mm}
    \cventry{ACM Richard Tapia Celebration of Diversity in Computing Conference Scholarship}{The National Science Foundation (NSF)}{Atlanta, GA}{Sep. 2017}{\begin{cvitems}
\item Included free conference registration, shared hotel accommodations and a \$500 travel stipend.
\end{cvitems}}
    \cventry{ACM Student Research Competition Travel Award}{Microsoft Research}{Atlanta, GA}{Sep. 2017}{\begin{cvitems}
\item \$500 travel stipend was awarded for ACM TAPIA 2017 Student Research Competition (SRC).
\end{cvitems}}
    \cventry{CSUN Conference Travel Award}{AccessComputing, University of Washington}{Seattle, WA}{Mar. 2017}{\begin{cvitems}
\item Sponsored by The National Science Foundation (NSF).
\end{cvitems}}
    \cventry{Summer Tuition Assistance Funding}{College of Education}{University Park, PA}{2017}{}\vspace{-4.0mm}
    \cventry{Graduate Student Travel Grant}{College of Education}{University Park, PA}{Aug. 2016 - Present}{}\vspace{-4.0mm}
    \cventry{Graduate Assistantship}{Teaching and Learning with Technology}{University Park, PA}{Aug. 2016 - Present}{}\vspace{-4.0mm}
    \cventry{Outstanding Academic Award}{Hyomyoung Scholarship}{Seoul, South Korea}{2014 - 2019}{}\vspace{-4.0mm}
    \cventry{All Nations Church Graduate Scholarship}{ANC Scholarship Foundation}{Lake View Terrace, CA}{2016}{}\vspace{-4.0mm}
\end{cventries}

\section{Service}\label{service}

\begin{cventries}
    \cventry{Search Committee for Director/Assistant Dean, Diversity, Equity, Inclusion, and Accessibility (DEIA)}{School of Information Sciences at the University of Illinois at Urbana-Champaign}{Champaign, IL}{April 2022 - Present}{}\vspace{-4.0mm}
    \cventry{Diversity Committee}{School of Information Sciences at the University of Illinois at Urbana-Champaign}{Champaign, IL}{Sep. 2021 - Apr. 2022}{}\vspace{-4.0mm}
    \cventry{Program Committee Member}{International Web for All Conference (W4A) 2022}{Virtual}{Dec. 2021 - Feb. 2022}{}\vspace{-4.0mm}
    \cventry{Accessibility Chair}{15th International Conference on Educational Data Mining}{Durham}{Oct. 2021 - Jul. 2022}{}\vspace{-4.0mm}
    \cventry{Journal Reviewer}{Special issue on ethnography and learning design}{TechTrends}{Sep. 2021}{}\vspace{-4.0mm}
    \cventry{Reviewer}{The Network of Academic Programs in the Learning Sciences (NAPLeS)}{International Society of the Learning Sciences}{2020}{\begin{cvitems}
\item Reviewed abstracts on equity and disability.
\end{cvitems}}
    \cventry{College of Education Technology Committee}{The Pennsylvania State University}{University Park, PA}{Aug. 2019 - May. 2020}{\begin{cvitems}
\item Worked as a student committee member to support understanding of educational technology for College of Education.
\end{cvitems}}
    \cventry{Libraries' Accessibility Student Advisory Group}{The Pennsylvania State University}{University Park, PA}{Aug. 2019 - May. 2020}{}\vspace{-4.0mm}
    \cventry{Reviewer}{The 14th International Conference of the Learning Sciences (ICLS)}{Nashville, TN}{2019}{}\vspace{-4.0mm}
    \cventry{Reviewer}{The ACM CHI Conference on Human Factors in Computing Systems (CHI 2020, 2021)}{Honolulu, HI, Yokohama, Japan}{2019, 2020}{}\vspace{-4.0mm}
    \cventry{Reviewer}{The 1st International Conference on Quantitative Ethnography (ICQE)}{Madison, WI}{2019}{}\vspace{-4.0mm}
    \cventry{Reviewer}{The 3rd Learning Sciences Graduate Student Conference (LSGSC)}{Nashville, TN}{2018}{}\vspace{-4.0mm}
    \cventry{Reviewer}{The 2nd Learning Sciences Graduate Student Conference (LSGSC)}{Bloomington, IN}{2017}{}\vspace{-4.0mm}
    \cventry{Funding Reviewer}{Center for Online Innovation in Learning (COIL), The Pennsylvania State University}{University Park, PA}{2016}{}\vspace{-4.0mm}
\end{cventries}

\section{Current Memberships}\label{current-memberships}

\begin{itemize}
\tightlist
\item
  International Society of the Learning Sciences (ISLS)
\item
  Association for Computing Machinery (ACM)
\item
  American Statistical Association (ASA)
\item
  International Society for Quantitative Ethnography (ISQE)
\item
  International Association of Accessibility Professionals (IAAP)
\item
  International Educational Data Mining Society (IEDMS)
\end{itemize}


\label{LastPage}~
\end{document}
