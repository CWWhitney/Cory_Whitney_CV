%!TEX TS-program = xelatex
%!TEX encoding = UTF-8 Unicode
% Awesome CV LaTeX Template for CV/Resume
%
% This template has been downloaded from:
% https://github.com/posquit0/Awesome-CV
%
% Author:
% Claud D. Park <posquit0.bj@gmail.com>
% http://www.posquit0.com
%
%
% Adapted to be an Rmarkdown template by Mitchell O'Hara-Wild
% 23 November 2018
%
% Template license:
% CC BY-SA 4.0 (https://creativecommons.org/licenses/by-sa/4.0/)
%
%-------------------------------------------------------------------------------
% CONFIGURATIONS
%-------------------------------------------------------------------------------
% A4 paper size by default, use 'letterpaper' for US letter
\documentclass[11pt, a4paper]{awesome-cv}

% Configure page margins with geometry
\geometry{left=1.4cm, top=.8cm, right=1.4cm, bottom=1.8cm, footskip=.5cm}

% Specify the location of the included fonts
\fontdir[fonts/]

% Color for highlights
% Awesome Colors: awesome-emerald, awesome-skyblue, awesome-red, awesome-pink, awesome-orange
%                 awesome-nephritis, awesome-concrete, awesome-darknight

\colorlet{awesome}{awesome-red}

% Colors for text
% Uncomment if you would like to specify your own color
% \definecolor{darktext}{HTML}{414141}
% \definecolor{text}{HTML}{333333}
% \definecolor{graytext}{HTML}{5D5D5D}
% \definecolor{lighttext}{HTML}{999999}

% Set false if you don't want to highlight section with awesome color
\setbool{acvSectionColorHighlight}{true}

% If you would like to change the social information separator from a pipe (|) to something else
\renewcommand{\acvHeaderSocialSep}{\quad\textbar\quad}

\def\endfirstpage{\newpage}

%-------------------------------------------------------------------------------
%	PERSONAL INFORMATION
%	Comment any of the lines below if they are not required
%-------------------------------------------------------------------------------
% Available options: circle|rectangle,edge/noedge,left/right

\name{JooYoung}{Seo}

\position{RStudio Intern for Summer 2020; Ph.D.~Candidate (ABD)}
\address{Learning, Design, and Technology, 301 Keller Building, University Park, PA 16802, USA}

\mobile{+1 814-777-5825}
\email{\href{mailto:jooyoung@psu.edu}{\nolinkurl{jooyoung@psu.edu}}}
\homepage{jooyoungseo.com}
\github{jooyoungseo}
\linkedin{jooyoungseo}
\twitter{sjystu}

% \gitlab{gitlab-id}
% \stackoverflow{SO-id}{SO-name}
% \skype{skype-id}
% \reddit{reddit-id}

\quote{I am a learning scientist, software engineer, and internationally-certified accessibility professional.}

\usepackage{booktabs}

% Templates for detailed entries
% Arguments: what when with where why
\usepackage{etoolbox}
\def\detaileditem#1#2#3#4#5{%
\cventry{#1}{#3}{#4}{#2}{\ifx#5\empty\else{\begin{cvitems}#5\end{cvitems}}\fi}\ifx#5\empty{\vspace{-4.0mm}}\else\fi}
\def\detailedsection#1{\begin{cventries}#1\end{cventries}}

% Templates for brief entries
% Arguments: what when with
\def\briefitem#1#2#3{\cvhonor{}{#1}{#3}{#2}}
\def\briefsection#1{\begin{cvhonors}#1\end{cvhonors}}

\providecommand{\tightlist}{%
	\setlength{\itemsep}{0pt}\setlength{\parskip}{0pt}}

%------------------------------------------------------------------------------




\begin{document}

% Print the header with above personal informations
% Give optional argument to change alignment(C: center, L: left, R: right)
\makecvheader

% Print the footer with 3 arguments(<left>, <center>, <right>)
% Leave any of these blank if they are not needed
% 2019-02-14 Chris Umphlett - add flexibility to the document name in footer, rather than have it be static Curriculum Vitae
\makecvfooter
  {March 19, 2020}
    {JooYoung Seo~~~·~~~Curriculum Vitae}
  {\thepage}


%-------------------------------------------------------------------------------
%	CV/RESUME CONTENT
%	Each section is imported separately, open each file in turn to modify content
%------------------------------------------------------------------------------



\hypertarget{education}{%
\section{Education}\label{education}}

\detailedsection{\detaileditem{Ph.D. in Learning, Design, and Technology}{Aug. 2016 - Present}{The Pennsylvania State University}{University Park, PA}{\item{Expected graduation: May 2021.}\item{Dissertation Title: ``Discovering Informal Learning Cultures of Blind Individuals Pursuing STEM Disciplines: A Quantitative Ethnography Using Public Listserv Archives.''}\item{Committee members: Drs. Gabriela T. Richard (adviser; dissertation chair), Roy B. Clariana, ChanMin Kim, and Mary Beth Rosson.}}\detaileditem{M.Ed. in Learning, Design, and Technology}{Aug. 2014 - May. 2016}{The Pennsylvania State University}{University Park, PA}{\item{GPA 3.97/4.0.}}\detaileditem{B.A. in Education}{Mar. 2009 - Feb. 2014}{Sungkyunkwan University}{Seoul, South Korea}{\item{GPA 4.08/4.50.}}\detaileditem{B.A. in English Literature}{Mar. 2009 - Feb. 2014}{Sungkyunkwan University}{Seoul, South Korea}{\item{GPA 4.08/4.50.}}}

\hypertarget{research-experience}{%
\section{Research Experience}\label{research-experience}}

\detailedsection{\detaileditem{Graduate Assistant}{Aug. 2016 - Present}{IT Accessibility Team, Teaching and Learning with Technology (TLT)}{University Park, PA}{\item{Developing “Image ALT Text for HTML” and “Image ALT Text for Word” badges using the Penn State digital badges platform.}\item{Evaluating suitability of digital badges according to web content accessibility guidelines (WCAG) 2.0.}\item{Testing and consulting for accessible HTML5/CSS3 webpages for Penn State University sites and learning management tools (e.g., Angel, Canvas, Voice Thread, Adobe products, etc.)}\item{Engaged in a project for future research on making visualized statistical data accessible.}\item{Developed a package for statistical software R, called tactileR, which converts visual data into touchable graphs in conjunction with a Swell Form Heating Machine.}}\detaileditem{Graduate Researcher}{Aug. 2016 - Present}{Playful Learning and Inclusive Design Research Group}{University Park, PA}{\item{Conducting interaction analysis and microethnographic studies for Dr. Gabriela Richard's inclusive gaming for learning and accessible makerspaces for youth with diverse background.}}\detaileditem{Principal Investigator}{May. 2019 - Present}{Online Interactions of the Blind Research Project}{University Park, PA}{\item{Conducting quantitative ethnography research on how blind learners pursue STEM disciplines as captured through a large-scale  mailing listservs.}\item{Using data science, computational linguistics (i.e., unsupervised machine learning for text mining; natural language processing) approaches coupled with conventional ethnographic methods.}}\detaileditem{Project Manager}{May. 2018 - Present}{Accessible RMarkdown Online Writer (AROW) Project, Teaching and Learning with Technology (TLT)}{University Park, PA}{\item{Developed an accessible web application for people with dis/abilities to easily compose a high-quality scientific document based on LAMP, AJAX, R Markdown, and MathML.}}\detaileditem{Principal Investigator}{Jun. 2017 - Present}{Accessibility of Maker Toolkits Research Project}{University Park, PA}{\item{Conducting usability and design research on how to make current electronics and maker toolkits more accessible for learners with visual impairments.}}\detaileditem{Site Manager}{Feb. 2015 - Dec. 2015}{Learning2CC Project}{University Park, PA}{\item{Participated in an ongoing learning community project for sharing tips for video caption for deaf learners.}\item{Developed Learning2CC, the online community site, using the WordPress platform.}}\detaileditem{Graduate Researcher}{Sep. 2014 - May. 2016}{Avenue PM Research Group}{University Park, PA}{\item{Participated in Dr. Simon Hooper's grant project sponsored by the U.S. Department of Education Stepping Stones Phase II program.}\item{Improved web accessibility for deaf-blind learners by removing blockers on website platform.}}\detaileditem{Student Researcher}{Feb. 2012 - Mar. 2012}{The Ministry of Knowledge Economy and Seoul National University}{Seoul, South Korea}{\item{Visited various places in California, such as CSUN conference, museums, universities, and other publiclocations, to observe universal design products and services.}\item{Produced a comparison report on universal design products and services in Korea and the United States.}\item{Presented the report to the Quality of Life Technology team at Seoul National University.}}}

\hypertarget{publications}{%
\section{Publications}\label{publications}}

\hypertarget{statistical-computing-software-developments}{%
\subsection{Statistical Computing Software Developments}\label{statistical-computing-software-developments}}

\begingroup
\setlength{\parindent}{-0.5in}
\setlength{\leftskip}{0.5in}

\hypertarget{refs_R_packages_integrated}{}
\leavevmode\hypertarget{ref-R-edmdown}{}%
\textbf{Seo, J.} (2020). \emph{Edmdown: Writing a reproducible article for journal of educational datamining in r markdown}. \url{https://github.com/jooyoungseo/edmdown}

\leavevmode\hypertarget{ref-R-jladown}{}%
\textbf{Seo, J.} (2020). \emph{Jladown: Writing a reproducible article for journal of learning analyticsin r markdown}. \url{https://github.com/jooyoungseo/jladown}

\leavevmode\hypertarget{ref-R-ezpickr}{}%
\textbf{Seo, J.}, \& Choi, S. (2019). \emph{Ezpickr: Easy data import using gui file picker and seamless communication between an excel and r}. \url{https://CRAN.R-project.org/package=ezpickr}

\leavevmode\hypertarget{ref-R-mboxr}{}%
\textbf{Seo, J.}, \& Choi, S. (2019). \emph{Mboxr: Reading, extracting, and converting an mbox file into a tibble}. \url{https://CRAN.R-project.org/package=mboxr}

\leavevmode\hypertarget{ref-R-youtubecaption}{}%
\textbf{Seo, J.}, \& Choi, S. (2019). \emph{Youtubecaption: Downloading youtube subtitle transcription in a tidy tibble data frame}. \url{https://CRAN.R-project.org/package=youtubecaption}

\endgroup

\hypertarget{refereed-journal-papers}{%
\subsection{Refereed Journal Papers}\label{refereed-journal-papers}}

\begingroup
\setlength{\parindent}{-0.5in}
\setlength{\leftskip}{0.5in}

\hypertarget{refs_journals}{}
\leavevmode\hypertarget{ref-seo2019maker}{}%
\textbf{Seo, J.} (2019). Is the maker movement inclusive of ANYONE?: Three accessibility considerations to invite blind makers to the making world. \emph{TechTrends}, \emph{63}(5), 514--520. \url{https://doi.org/10.1007/s11528-019-00377-3}

\leavevmode\hypertarget{ref-seo2019arow}{}%
\textbf{Seo, J.}, \& McCurry, S. (2019). LaTeX is not easy: Creating accessible scientific documents with r markdown. \emph{Journal on Technology and Persons with Disabilities}, \emph{7}, 157--171.

\endgroup

\hypertarget{papers-in-refereed-conference-proceedings}{%
\subsection{Papers in Refereed Conference Proceedings}\label{papers-in-refereed-conference-proceedings}}

\begingroup
\setlength{\parindent}{-0.5in}
\setlength{\leftskip}{0.5in}

\hypertarget{refs_proceedings}{}
\leavevmode\hypertarget{ref-seo2020coding}{}%
\textbf{Seo, J.}, \& Richard, G. T. (in press-). Coding through touch: Exploring and re-designing tactile making activities with learners with visual dis/abilities. In M. Gresalfi \& I. Horn (Eds.), \emph{Interdisciplinarity in the learning sciences, 14th international conference of the learning sciences (icls) 2020}. International Society of the Learning Sciences (ISLS).

\leavevmode\hypertarget{ref-seo2018making}{}%
\textbf{Seo, J.} (2018). Accessibility and inclusivity in making: Engaging learners with all abilities in making activities. In \emph{Proceedings of the 3rd learning sciences graduate student conference} (pp. 141--142). LSGSC Planning Team.

\leavevmode\hypertarget{ref-seo2018accessibility}{}%
\textbf{Seo, J.}, \& Richard, G. T. (2018). Accessibility, making and tactile robotics: Facilitating collaborative learning and computational thinking for learners with visual impairments. In J. Kay \& R. Luckin (Eds.), \emph{Rethinking learning in the digital age: Making the learning sciences count, 13th international conference of the learning sciences (icls) 2018} (Vol. 3, pp. 1755--1757). International Society of the Learning Sciences (ISLS).

\leavevmode\hypertarget{ref-konecki2017role}{}%
Konecki, M., Lovrenčić, S., \textbf{Seo, J.}, \& LaPierre, C. (2017). The role of ict in aiding visually impaired students and professionals. \emph{Proceedings of the 11th Multidisciplinary Academic Conference}, 148.

\leavevmode\hypertarget{ref-seo2017embracing}{}%
\textbf{Seo, J.}, AlQahtani, M., Ouyang, X., \& Borge, M. (2017). Embracing learners with visual impairments in cscl. In B. K. Smith, M. Borge, E. Mercier, \& K. Y. Lim (Eds.), \emph{Making a difference: Prioritizing equity and access in cscl, 12th international conference on computer supported collaborative learning (cscl) 2017} (Vol. 2). International Society of the Learning Sciences (ISLS).

\leavevmode\hypertarget{ref-seo2019discovering}{}%
\textbf{Seo, J.} (2019). Discovering informal learning cultures of blind individuals pursuing stem disciplines: A quantitative ethnography using listserv archives. \emph{The First International Conference on Quantitative Ethnography: Doctoral Consortium}, S66--S67.

\endgroup

\hypertarget{publications-in-healthcare}{%
\subsection{Publications in Healthcare}\label{publications-in-healthcare}}

\begin{itemize}
\tightlist
\item
  I have contributed to the following publications for their methods and research designs.
\end{itemize}

\begingroup
\setlength{\parindent}{-0.5in}
\setlength{\leftskip}{0.5in}

\hypertarget{refs_healthcare}{}
\leavevmode\hypertarget{ref-mentalhealth}{}%
Choi, S., \& \textbf{Seo, J.} (in press-). An exploratory study of the research on caregiver depression: Using bibliometrics and lda topic modeling. \emph{Issues in Mental Health Nursing}. \url{https://doi.org/10.1080/01612840.2019.1705944}

\leavevmode\hypertarget{ref-doi:10.1111ux2fnuf.12328}{}%
Choi, S., \& \textbf{Seo, J.} (2019). Analysis of caregiver burden in palliative care: An integrated review. \emph{Nursing Forum}, \emph{54}(2), 280--290. \url{https://doi.org/10.1111/nuf.12328}

\leavevmode\hypertarget{ref-choi2019heart}{}%
Choi, S., \& \textbf{Seo, J.} (2019). Heart failure research using text mining: A systematic review. \emph{NURSING RESEARCH}, \emph{68}(2), E119--E120.

\leavevmode\hypertarget{ref-choi2019trends}{}%
Choi, S., \& \textbf{Seo, J.} (2019). Trends in self-management among adults with heart failure from 2011 to 2016 using the national health and nutrition examination surveys. \emph{Journal of Cardiac Failure}, \emph{25}(8), S4. \url{https://doi.org/10.1016/j.cardfail.2019.07.540}

\leavevmode\hypertarget{ref-choi2018effects}{}%
Choi, S., \& \textbf{Seo, J.} (2018). Effects of nonpharmacological interventions for fatigue in patients with heart failure: A systematic review and meta-analysis. \emph{Journal of Cardiac Failure}, \emph{24}(8), S73--S74.

\leavevmode\hypertarget{ref-choi2019exploring}{}%
Choi, S., \textbf{Seo, J.}, \& Kitko, L. (2019). Exploring the lived experience of a family member with advanced heart failure: Using a text mining approach. \emph{The First International Conference on Quantitative Ethnography: Poster Session}, S8--S9.

\endgroup

\hypertarget{current-memberships}{%
\section{Current Memberships}\label{current-memberships}}

\begin{itemize}
\tightlist
\item
  Member, International Society of the Learning Sciences (ISLS)
\item
  Member, Association for Computing Machinery (ACM)
\item
  Member, American Educational Research Association (AERA)
\item
  Member, Association for Educational Communications and Technology (AECT)
\item
  Member, International Association of Accessibility Professionals (IAAP)
\end{itemize}

\end{document}
