%!TEX TS-program = xelatex
%!TEX encoding = UTF-8 Unicode
% Awesome CV LaTeX Template for CV/Resume
%
% This template has been downloaded from:
% https://github.com/posquit0/Awesome-CV
%
% Author:
% Claud D. Park <posquit0.bj@gmail.com>
% http://www.posquit0.com
%
%
% Adapted to be an Rmarkdown template by Mitchell O'Hara-Wild
% 23 November 2018
%
% Template license:
% CC BY-SA 4.0 (https://creativecommons.org/licenses/by-sa/4.0/)
%
%-------------------------------------------------------------------------------
% CONFIGURATIONS
%-------------------------------------------------------------------------------
% A4 paper size by default, use 'letterpaper' for US letter
\documentclass[11pt,a4paper,]{awesome-cv}

% Configure page margins with geometry
\usepackage{geometry}
\geometry{left=1.4cm, top=.8cm, right=1.4cm, bottom=1.8cm, footskip=.5cm}


% Specify the location of the included fonts
\fontdir[fonts/]

% Color for highlights
% Awesome Colors: awesome-emerald, awesome-skyblue, awesome-red, awesome-pink, awesome-orange
%                 awesome-nephritis, awesome-concrete, awesome-darknight

\definecolor{awesome}{HTML}{414141}

% Colors for text
% Uncomment if you would like to specify your own color
% \definecolor{darktext}{HTML}{414141}
% \definecolor{text}{HTML}{333333}
% \definecolor{graytext}{HTML}{5D5D5D}
% \definecolor{lighttext}{HTML}{999999}

% Set false if you don't want to highlight section with awesome color
\setbool{acvSectionColorHighlight}{true}

% If you would like to change the social information separator from a pipe (|) to something else
\renewcommand{\acvHeaderSocialSep}{\quad\textbar\quad}

\def\endfirstpage{\newpage}

%-------------------------------------------------------------------------------
%	PERSONAL INFORMATION
%	Comment any of the lines below if they are not required
%-------------------------------------------------------------------------------
% Available options: circle|rectangle,edge/noedge,left/right

\name{JooYoung}{Seo}

\position{Ph.D.}
\address{614 E. Daniel St.~\textbar{} Room 5158, Champaign, IL 61820}

\mobile{+1 217-333-2671}
\email{\href{mailto:jseo1005@illinois.edu}{\nolinkurl{jseo1005@illinois.edu}}}
\homepage{jooyoungseo.github.io}
\github{jooyoungseo}
\linkedin{jooyoungseo}
\twitter{seo\_jooyoung}

% \gitlab{gitlab-id}
% \stackoverflow{SO-id}{SO-name}
% \skype{skype-id}
% \reddit{reddit-id}

\quote{I am a learning scientist, data-science/software engineer, and
internationally certified accessibility professional.}

\usepackage{booktabs}

\providecommand{\tightlist}{%
	\setlength{\itemsep}{0pt}\setlength{\parskip}{0pt}}

%------------------------------------------------------------------------------


\usepackage{fancyhdr}
\pagestyle{fancy}
\fancyhf{}
\fancyhead[R]{\thepage}

% Pandoc CSL macros
\newlength{\cslhangindent}
\setlength{\cslhangindent}{1.5em}
\newlength{\csllabelwidth}
\setlength{\csllabelwidth}{3em}
\newenvironment{CSLReferences}[3] % #1 hanging-ident, #2 entry spacing
 {% don't indent paragraphs
  \setlength{\parindent}{0pt}
  % turn on hanging indent if param 1 is 1
  \ifodd #1 \everypar{\setlength{\hangindent}{\cslhangindent}}\ignorespaces\fi
  % set entry spacing
  \ifnum #2 > 0
  \setlength{\parskip}{#2\baselineskip}
  \fi
 }%
 {}
\usepackage{calc}
\newcommand{\CSLBlock}[1]{#1\hfill\break}
\newcommand{\CSLLeftMargin}[1]{\parbox[t]{\csllabelwidth}{#1}}
\newcommand{\CSLRightInline}[1]{\parbox[t]{\linewidth - \csllabelwidth}{#1}}
\newcommand{\CSLIndent}[1]{\hspace{\cslhangindent}#1}

\begin{document}

% Print the header with above personal informations
% Give optional argument to change alignment(C: center, L: left, R: right)
\makecvheader

% Print the footer with 3 arguments(<left>, <center>, <right>)
% Leave any of these blank if they are not needed
% 2019-02-14 Chris Umphlett - add flexibility to the document name in footer, rather than have it be static Curriculum Vitae
\makecvfooter
{March 18, 2022}
{JooYoung Seo~~~·~~~Curriculum Vitae}
{\thepage}


%-------------------------------------------------------------------------------
%	CV/RESUME CONTENT
%	Each section is imported separately, open each file in turn to modify content
%------------------------------------------------------------------------------


\hypertarget{professional-appointment}{%
    \section{Professional Appointment}\label{professional-appointment}}

\begin{cventries}
    \cventry{Assistant Professor}{School of Information Sciences}{University of Illinois at Urbana-Champaign}{Sep. 2021 - Present}{}\vspace{-4.0mm}
    \cventry{Faculty Affiliate}{National Center for Supercomputing Applications}{University of Illinois at Urbana-Champaign}{Dec. 2021 - Present}{}\vspace{-4.0mm}
    \cventry{Faculty Affiliate}{Illinois Informatics Institute}{University of Illinois at Urbana-Champaign}{Sep. 2021 - Present}{}\vspace{-4.0mm}
    \cventry{Faculty Affiliate}{IDEA (Inclusion, Diversity, Equity, and Access) Institute}{University of Illinois at Urbana-Champaign}{Jan. 2022 - Present}{}\vspace{-4.0mm}
    \cventry{Software Engineer Intern}{RStudio}{Boston, MA}{May. 2020 - Aug. 2020}{}\vspace{-4.0mm}
\end{cventries}

\hypertarget{education}{%
    \section{Education}\label{education}}

\begin{cventries}
    \cventry{Ph.D. in Learning, Design, and Technology}{The Pennsylvania State University}{University Park, PA}{2021}{\begin{cvitems}
            \item Dissertation Title: ``Discovering Informal Learning Cultures of Blind Individuals Pursuing STEM Disciplines: A Computational Ethnography Using Public Listserv Archives.''
            \item Committee members: Drs. Gabriela T. Richard (adviser; dissertation chair), Roy B. Clariana, ChanMin Kim, and Mary Beth Rosson.
        \end{cvitems}}
    \cventry{M.Ed. in Learning, Design, and Technology}{The Pennsylvania State University}{University Park, PA}{2016}{}\vspace{-4.0mm}
    \cventry{Double B.A. in Education, English Literature}{Sungkyunkwan University}{Seoul, South Korea}{2014}{}\vspace{-4.0mm}
\end{cventries}

\hypertarget{selected-publications}{%
    \section{Selected Publications}\label{selected-publications}}

\hypertarget{refs_selected}{}
\leavevmode\vadjust pre{\hypertarget{ref-seo2021csun}{}}%
\textbf{Seo, J.}, \& Choi, S. (in press). Are blind people considered a
part of scientific knowledge producers?: Accessibility report on top-10
SCIE journal systems using a tripartite evaluation approach.
\emph{Journal on Technology and Persons with Disabilities}.

\leavevmode\vadjust pre{\hypertarget{ref-seoScopingReviewThree2022}{}}%
\textbf{Seo, J.}, Moon, J., Choi, G. W., \& Do, J. (2022). A {Scoping
        Review} of {Three Computational Approaches} to {Ethnographic Research}
in {Digital Learning Environments}. \emph{TechTrends}, \emph{66}(1),
102--111. \url{https://doi.org/10.1007/s11528-021-00689-3}

\leavevmode\vadjust pre{\hypertarget{ref-seoSCAFFOLDingAllAbilities2021a}{}}%
\textbf{Seo, J.}, \& Richard, G. T. (2021). {SCAFFOLDing All Abilities}
into {Makerspaces}: {A Design Framework} for {Universal}, {Accessible}
and {Intersectionally Inclusive Making} and {Learning}.
\emph{Information and Learning Sciences}, \emph{122}(11/12), 795--815.
\url{https://doi.org/10.1108/ILS-10-2020-0230}

\leavevmode\vadjust pre{\hypertarget{ref-seo2020coding}{}}%
\textbf{Seo, J.}, \& Richard, G. T. (2020). Coding through touch:
Exploring and re-designing tactile making activities with learners with
visual dis/abilities. In M. Gresalfi \& I. Horn (Eds.),
\emph{Interdisciplinarity in the learning sciences, 14th international
    conference of the learning sciences (ICLS) 2020} (Vol. 3, pp.
1373--1380). Nashville, TN: International Society of the Learning
Sciences (ISLS).

\leavevmode\vadjust pre{\hypertarget{ref-seo2019discovering}{}}%
\textbf{Seo, J.} (2019). Discovering informal learning cultures of blind
individuals pursuing STEM disciplines: A quantitative ethnography using
listserv archives. \emph{The first international conference on
    quantitative ethnography: Doctoral consortium}, S66--S67. Madison, WI.
\emph{Awarded the best Doctoral Consortium Proposal Cengage fellowship}.

\leavevmode\vadjust pre{\hypertarget{ref-seo2019maker}{}}%
\textbf{Seo, J.} (2019). Is the maker movement inclusive of {ANYONE}?:
Three accessibility considerations to invite blind makers to the making
world. \emph{{TechTrends}}, \emph{63}(5), 514--520.
\url{https://doi.org/10.1007/s11528-019-00377-3}

\leavevmode\vadjust pre{\hypertarget{ref-seo2019arow}{}}%
\textbf{Seo, J.}, \& McCurry, S. (2019). LaTeX is NOT easy: Creating
accessible scientific documents with r markdown. \emph{Journal on
    Technology and Persons with Disabilities}, \emph{7}, 157--171.

\leavevmode\vadjust pre{\hypertarget{ref-seo2018making}{}}%
\textbf{Seo, J.} (2018). Accessibility and inclusivity in making:
Engaging learners with all abilities in making activities. In
\emph{Proceedings of the 3rd learning sciences graduate student
    conference} (pp. 141--142). Nashville, TN: LSGSC Planning Team.

\leavevmode\vadjust pre{\hypertarget{ref-seo2018accessibility}{}}%
\textbf{Seo, J.}, \& Richard, G. T. (2018). Accessibility, making and
tactile robotics: Facilitating collaborative learning and computational
thinking for learners with visual impairments. In J. Kay \& R. Luckin
(Eds.), \emph{Rethinking learning in the digital age: Making the
    learning sciences count, 13th international conference of the learning
    sciences (ICLS) 2018} (Vol. 3, pp. 1755--1757). London, UK:
International Society of the Learning Sciences (ISLS).

\leavevmode\vadjust pre{\hypertarget{ref-seo2017embracing}{}}%
\textbf{Seo, J.}, AlQahtani, M., Ouyang, X., \& Borge, M. (2017).
Embracing learners with visual impairments in CSCL. In B. K. Smith, M.
Borge, E. Mercier, \& K. Y. Lim (Eds.), \emph{Making a difference:
    Prioritizing equity and access in CSCL, 12th international conference on
    computer supported collaborative learning (CSCL) 2017} (Vol. 2, pp.
573--576). Philadelphia, PA: International Society of the Learning
Sciences (ISLS).

\hypertarget{grants}{%
    \section{Grants}\label{grants}}

\begin{cventries}
    \cventry{Sponsored Research Funding}{RStudio PBC}{Boston, MA}{May. 2022 - Apr. 2023}{\begin{cvitems}
            \item Improving Accessible Reproducibility for Data Science Publishing System
            \item Funding amount: \$54,327.00.
        \end{cvitems}}
    \cventry{Wallace Foundation grant}{Emerging Scholars Program}{International Society of the Learning Sciences}{Jan. 2022 - Dec. 2022}{\begin{cvitems}
            \item Data Accessibilization: Making Data Science Education Accessible for Blind Learners
            \item Funding amount: \$10,000.00.
        \end{cvitems}}
    \cventry{Dissertation Research Initiation Grant}{College of Education, The Pennsylvania State University}{University Park, PA}{Feb. 2020 - Dec. 2020}{\begin{cvitems}
            \item Selected as one of the 10 outstanding dissertation research proposals in 2019-2020 academic year.
            \item Funding amount: \$600.00 USD.
        \end{cvitems}}
    \cventry{Doctoral Research}{ChungInwook Scholarship Foundation}{Seoul, South Korea}{2016 - 2018}{\begin{cvitems}
            \item Funding amount: \$30,000.00 USD.
        \end{cvitems}}
    \cventry{Innovative Research Funding}{Center for Online Innovation in Learning, The Pennsylvania State University}{University Park, PA}{Mar. 2016 - Aug. 2017}{\begin{cvitems}
            \item Providing Touchable Knowledge Structure Graphic Feedback for Blind Online Learners: Tablet-Based Haptic Feedback and Paper-Based Tactile Feedback
            \item Developed a prototype website and mobile app for accessible knowledge structure.
            \item Co-led (Co-PI) a development project funded by Penn State to provide blind online learners with accessible network graphs that measure students’ knowledge structure.
            \item Funding amount: \$33,060.00 USD.
        \end{cvitems}}
    \cventry{Future Interdisciplinary Study}{National Institute for International Education, Korean Government}{Seoul, South Korea}{2014 - 2016}{\begin{cvitems}
            \item Fully sponsored by Korean Government for the Master's research.
            \item Funding amount: \$70,000.00 USD.
        \end{cvitems}}
\end{cventries}


\end{document}
